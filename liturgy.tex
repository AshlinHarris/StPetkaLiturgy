
% ==============================================================================
% NOTES
% ==============================================================================

%TODO: make big text version

%TODO: updating .ily template doesn't cause hymns to be remade
%TODO: I have to manually delete them

%TODO: which phrases are only extended when the priest serves alone?
%TODO: find sources that can be freely shared
%TODO: transcribe Vespers chants
%TODO: can page numbers be placed higher up on the page?
%TODO: solve problem of pagebreaking consistently in general
%TODO: add accents consistently throughout
%TODO: include silent prayers?

%TODO: notice that Cyrillic boolean does not affect how latex music is produced
%TODO: can I combine the makefiles for the liturgy book and the hymns?

% Cyrillic text: http://www.fatheralexander.org/booklets/russian/liturg.htm

% ==============================================================================
% OPTIONS
% ==============================================================================

%\batchmode % do not prompt on warnings

\newif\ifCyrillic
\Cyrillictrue % use Cyrillic characters for Slavonic and Serbian text (Choir only)
%\Cyrillicfalse

%TODO: I have not used this option, so it is untested
\newif\ifIncludeMusic
\IncludeMusictrue
%\IncludeMusicfalse

%TODO: variable for including spoken parts?

% ==============================================================================
% PREAMBLE
% ==============================================================================

\documentclass[twoside]{article}

\usepackage{xifthen} % \isempty{}

% Adjust margin sizes
\usepackage[margin=0.5in]{geometry} % smaller margins, more text

% Support Cyrillic characters
\pdfinclusioncopyfonts=1 % Cyrillic support for included PDFs
\usepackage{cmap} % allow PDF readers to search for Cyrillic characters
\usepackage[T1, T2A]{fontenc}
\usepackage[utf8]{inputenc}
\usepackage[main=english, russian]{babel}

\sloppy % prevent line overhang in the margins

% section title format
\usepackage{titlesec}
\titleformat{\section}[display]{\normalfont\bfseries}{}{0pt}{\Huge}

% disable numbering for subsections, etc.
\makeatletter
\renewcommand{\@seccntformat}[1]{}
\makeatother

% fix page references to the first page of included PDFs
% otherwise, labels will refer to the last page
\makeatletter
\newcommand{\HymnFullPage}[1]{
	\clearpage
\label{#1}
\includepdf[pages=-, pagecommand={}]{HYMNS_DIRECTORY/#1}
\immediate\write\@auxout{\string\newlabel{#1-lastpage}{{\@currentlabel}{\number\numexpr\value{page}-1}}}
}
\makeatother

% negate indentation from lilypond quote
% (this is required to keep fragments printed completely on page)
\newenvironment{HymnPartPage}{\list{}{\leftmargin=-1.0cm\rightmargin=0.0cm}\item[]}{\endlist}

	% no headers (or footers) - keep page numbers
	\pagestyle{plain}

	% no section numbers
\renewcommand\thesection{}
\renewcommand\thesubsection{}
\renewcommand\thesubsubsection{}

% ------------------------------------------------------------------------------
% Miscellaneous Packages
% ------------------------------------------------------------------------------

\usepackage{comment} % block comment environment
\usepackage{paracol} % multiple columns
\usepackage{parskip} % do not indent paragraphs
\usepackage{pdfpages} % include pdf files as complete pages


% ==============================================================================
% COMMANDS
% ==============================================================================

% ------------------------------------------------------------------------------
% Basic Formats
% ------------------------------------------------------------------------------

\newcommand{\LITMOD}  [1]{\textbf{#1}} % modifiers such as (3x) and (extended)
\newcommand{\LITSPK}  [1]{\textbf{#1}} % speakers such as Priest and Deacon
\newcommand{\LITTITLE}[1]{\textbf{#1}} % titles of hymns and the Creed
\newcommand{\LITSUB}  [1]{\textit{#1}} % subtitles of hymns and the Creed
\newcommand{\LITHYMN} [1]{#1}          % words of hymns and the Creed
\newcommand{\LITNOTE} [1]{\textit{#1}} % explanatory notes
\newcommand{\LITALT}  [1]{\textit{#1}} % alternatives and placeholders
\newcommand{\LITOPT}  [1]{\textit{#1}} % optional or variable passages

% ------------------------------------------------------------------------------
% Speaker Commands
% ------------------------------------------------------------------------------

\newcommand{\LiturgyComment}[3]{\textit{
	#3
%	\begin{paracol}{2}
%		{\selectlanguage{russian}#1}
%			\switchcolumn#3
%	\end{paracol}
}}

\newcommand{\LiturgyHeader}[3]{\LITSPK{
	\begin{paracol}{2}
		{\selectlanguage{russian}#1}
			\switchcolumn#3
	\end{paracol}
}}

\newcommand{\LiturgyReference}[3]{\LITSPK{
	\begin{paracol}{2}
		{\selectlanguage{russian}(#1)}
			\switchcolumn(#3)
	\end{paracol}
}}

\newcommand{\DeaconSays}[3]{
	\begin{paracol}{2}
		{\selectlanguage{russian}\LITSPK{Диакон:} #1}
		\switchcolumn\LITSPK{Deacon:} #3
	\end{paracol}
}

\newcommand{\PriestSays}[3]{%
	\begin{paracol}{2}
		{\selectlanguage{russian}\LITSPK{Иерей:} #1}
		\switchcolumn\LITSPK{Priest:} #3
	\end{paracol}
}

\newcommand{\ChoirSays}[3]{
	\begin{paracol}{2}
		\ifCyrillic{\selectlanguage{russian}\LITSPK{Лик:} #1}
		\else\LITSPK{Choir:} #2\fi
		\switchcolumn\LITSPK{Choir:} #3
	\end{paracol}
}

	\newcommand{\ChoirSings}[3]{
		\begin{paracol}{2}
			\ifCyrillic{\selectlanguage{russian}\LITSPK{Лик:} \LITHYMN{#1}}
			\else\LITSPK{Choir:} \LITHYMN{#2}\fi
			\switchcolumn\LITSPK{Choir:} \LITHYMN{#3}
		\end{paracol}
	}

\newcommand{\ReaderSays}[3]{
	\begin{paracol}{2}
		{\selectlanguage{russian}\LITSPK{Чмец:} #1}
		\switchcolumn\LITSPK{Reader:} #3
	\end{paracol}
}

\newcommand{\PeopleSay}[3]{
	\begin{paracol}{2}
		\ifCyrillic{\selectlanguage{russian}\LITSPK{Људи:} #1}
		\else\LITSPK{Људи:} #2\fi
		\switchcolumn\LITSPK{People:} #3
	\end{paracol}
}

\newcommand{\IbidSays}[3]{
	\begin{paracol}{2}
		\ifCyrillic{\selectlanguage{russian}#1}
%TODO: check other strings for being empty, like so:
\else\ifthenelse{\isempty{#2}}{{\selectlanguage{russian}#1}}{#2}\fi
		\switchcolumn#3
	\end{paracol}
}

\newcommand{\IbidChoirSays}[3]{
	\begin{paracol}{2}
		\ifCyrillic{\selectlanguage{russian}#1}
		\else#2\fi
		\switchcolumn#3
	\end{paracol}
}

	\newcommand{\IbidSings}[3]{
		\begin{paracol}{2}
			\ifCyrillic{\selectlanguage{russian}\LITHYMN{#1}}
			\else\LITHYMN{#2}\fi
			\switchcolumn\LITHYMN{#3}
		\end{paracol}
	}

% ==============================================================================
% DOCUMENT
% ==============================================================================

\date{}

\begin{document}

\begin{comment}
\title{Divine Liturgy Hymnal}
\maketitle
\clearpage
\end{comment}

%\tableofcontents

\begin{comment}
БОЖЕСТВЕННАЯ ЛИТУРГИЯ
ИЖЕ ВО СВЯТЫХ ОТЦА НАШЕГО
ИОАННА ЗЛАТОУСТАГО.
Времени же наставшу, входит священник во храм и соединився со диаконом, творят вкупе к востоку пред святыми дверьми поклонения три и молитвы. Вшедше же во святилище облачатся и творят проскомидию. По отпусте же проскомидии кадит диакон святое предложение, святилище и храм весь, входит паки во святый олтарь, и покадив святую трапезу паки, и священника, кадильницу убо отлагает на место свое, сам же приходит ко иерею. И ставше вкупе пред святою трапезою, покланяются трижды, в себе молящеся и глаголюще:
Цар\'{ю} Неб\'{е}сный, Ут\'{е}шителю, Д\'{у}ше \'{и}стины, И́же везд\'{е} сый и вся исполн\'{я}яй, Сокр\'{о}вище благ\'{и}х и ж\'{и}зни Под\'{а}телю, приид\'{и} и всел\'{и}ся в ны, и оч\'{и}сти ны от вс\'{я}кия скв\'{е}рны, и спас\'{и}, Бл\'{а}же, д\'{у}ши н\'{а}ша.
Сл\'{а}ва в В\'{ы}шних Б\'{о}гу, и на земл\'{и} мир, в челов\'{е}цех благовол\'{е}ние. Дважды.
Г\'{о}споди, устн\'{е} мо\'{и} отв\'{е}рзеши, и уст\'{а} мо\'{я} возвест\'{я}т хвал\'{у} Тво\'{ю}.
Таже целуют, священник убо святое Евангелие, диакон же святую трапезу. И посем подклонив диакон свою главу священнику, держа и орарь треми персты десныя руки, глаголет:
Вр\'{е}мя сотвор\'{и}ти Г\'{о}сподеви, влад\'{ы}ко, благослов\'{и}.
Священник, знаменуя его, глаголет: Благослов\'{е}н Бог наш всегд\'{а}, н\'{ы}не и пр\'{и}сно и во в\'{е}ки век\'{о}в.
Таже диакон: Помол\'{и}ся о мн\'{е}, влад\'{ы}ко свят\'{ы}й.
Священник: Да испр\'{а}вит Госп\'{о}дь стоп\'{ы} Тво\'{я}.
И паки диакон: Помян\'{и} мя, влад\'{ы}ко свят\'{ы}й.
Священник: Да помян\'{е}т тя Госп\'{о}дь Бог во Ц\'{а}рствии Сво\'{е}м всегд\'{а}, н\'{ы}не и пр\'{и}сно и во в\'{е}ки век\'{о}в.
Диакон же: Ам\'{и}нь.
И поклонився исходит северными дверьми, понеже царския двери до входа не отверзаются. И став на обычном месте, прямо святых дверей, покланяется со благоговением, трижды, глаголя в себе:
Г\'{о}споди, устн\'{е} мо\'{и} отв\'{е}рзеши, и уст\'{а} моя возвест\'{я}т хвал\'{у} Тво\'{ю}.
\end{comment}

\section{Divine Liturgy}

\DeaconSays{Благослов\'{и}, Влад\'{ы}ко.}{}{Bless, Master.}
\PriestSays{Благослов\'{е}но Ц\'{а}рство Отц\'{а} и С\'{ы}на и Свят\'{а}го Д\'{у}ха, н\'{ы}не и пр\'{и}сно и во в\'{е}ки век\'{о}в.}{}{Blessed is the Kingdom of the Father, and of the Son, and of the Holy Spirit, now and ever and unto ages of ages.}
\ChoirSays{Ам\'{и}нь.}{Amin.}{Amen.}
\LITNOTE{From Pascha until Ascension, after Amen, the Paschal Tropar (``Christ is risen\ldots'') is sung three times.}
\ifIncludeMusic
%TODO: I think if I use \lilypondfile, I must erase all the temporary files so that the version in the book will be refreshed
\begin{HymnPartPage}\lilypondfile[quote, noindent, line-width=8.000000\in]{HYMNS_DIRECTORY/Paschal_Tropar_Long.sub.ly}\end{HymnPartPage}
\else
\ChoirSings{Христос воскресе из мертвых, смертию смерть поправ, и сущим во гробех живот даровав!}{Hristos voskrjesje iz mjertvyh, smjertiju smjert poprav, i suŝim vo grobjeh život darovav!}{Christ is risen from the dead, trampling down death by death, and upon those in the tombs bestowing life!}
\fi
%	\clearpage % keeps next hymn on a single page to avoid a bad page reference
%\begin{HymnPartPage}\lilypondfile[quote, noindent, line-width=8.000000\in]{HYMNS_DIRECTORY/Paschal_Tropar_Short.sub.ly}\end{HymnPartPage}

\clearpage
\LiturgyHeader{Великая ектения}{}{Great Litany}
%\subsection{Great Litany}

\DeaconSays{М\'{и}ром Г\'{о}споду пом\'{о}лимся.}{}{In peace let us pray to the Lord.}
\ChoirSays{Г\'{о}споди, пом\'{и}луй.}{Gospodi, pomiluj.}{Lord, have mercy.}
%TODO: another refrain? (for the salvation of our souls...)
\DeaconSays{О св\'{ы}шнем м\'{и}ре и спас\'{е}нии душ н\'{а}ших, Г\'{о}споду пом\'{о}лимся.}{}{For the peace of the whole world, for the welfare of the holy churches of God, and for the unity of all, let us pray to the Lord.}
\ChoirSays{Г\'{о}споди, пом\'{и}луй.}{Gospodi, pomiluj.}{Lord, have mercy.}
%TODO: are these said? This is from Leonidas' copy
%\DeaconSays{О м\'{и}ре всег\'{о} мира, благосто\'{я}нии свят\'{ы}х Б\'{о}жиих церкв\'{е}й и соедин\'{е}нии всех, Г\'{о}споду пом\'{о}лимся.}{}{}
%\ChoirMercy
\DeaconSays{О свят\'{е}м хр\'{а}ме сем и с в\'{е}рою, благогов\'{е}нием и стр\'{а}хом Б\'{о}жиим вход\'{я}щих в онь, Г\'{о}споду пом\'{о}лимся.}{}{For this holy house and for those who enter with faith, reverence, and the fear of God, let us pray to the Lord.}
\ChoirSays{Г\'{о}споди, пом\'{и}луй.}{Gospodi, pomiluj.}{Lord, have mercy.}
%TODO: О Вел\'{и}ком Господ\'{и}не и отц\'{е} н\'{а}шем, Свят\'{е}йшем Патри\'{а}рхе (имярек), и о господ\'{и}не н\'{а}шем преосвящ\'{е}ннейшем митропол\'{и}те (или архиеп\'{и}скопе, или еп\'{и}скопе) (имярек), честн\'{е}м пресв\'{и}терстве, во Христ\'{е} ди\'{а}констве и о всем пр\'{и}чте и л\'{ю}дех, Г\'{о}споду пом\'{о}лимся.
%\ChoirMercy
\DeaconSays{О святейшем патриарсе нашем Иринеј, преосвященнейшем епископе Лонгин, честнем пресвитерстве, во Христе диаконстве, о всем притче и людех Господу помолимся.}{}{For our Most Holy Patriarch Irinej, for our Right Reverend Bishop Longin, for the honourable priesthood, the diaconate in Christ, for all the clergy and the people, let us pray to the Lord.}
\ChoirSays{Г\'{о}споди, пом\'{и}луй.}{Gospodi, pomiluj.}{Lord, have mercy.}
%TODO:
%O Presidentje strani seja, o praviteljstvje i voinstvje nasem, Gospodu pomolimsja.
%For the President of our country, for all civil authorities and the armed forces, let us pray to the Lord.
\DeaconSays{О Богохран\'{и}мей стран\'{е} н\'{а}шей, власт\'{е}х и в\'{о}инстве е\'{я}, Г\'{о}споду пом\'{о}лимся.}{}{For this land, its authorities and Armed Forces, let us pray to the Lord.}
\ChoirSays{Г\'{о}споди, пом\'{и}луй.}{Gospodi, pomiluj.}{Lord, have mercy.}
\DeaconSays{О гр\'{а}де сем, (или о в\'{е}си сей, или о свят\'{е}й об\'{и}тели сей), вс\'{я}ком гр\'{а}де, стран\'{е} и в\'{е}рою жив\'{у}щих в н\'{и}х, Г\'{о}споду пом\'{о}лимся.}{}{For this city, for every city and country, and for the faithful dwelling in them, let us pray to the Lord.}
\ChoirSays{Г\'{о}споди, пом\'{и}луй.}{Gospodi, pomiluj.}{Lord, have mercy.}
\DeaconSays{О бл\'{а}гораствор\'{е}нии возд\'{у}хов, о изоб\'{и}лии плод\'{о}в земн\'{ы}х и вр\'{е}менех м\'{и}рных, Г\'{о}споду пом\'{о}лимся.}{}{For seasonable weather, for abundance of the fruits of the earth, and for peaceful times, let us pray to the Lord.}
\ChoirSays{Г\'{о}споди, пом\'{и}луй.}{Gospodi, pomiluj.}{Lord, have mercy.}
\DeaconSays{О пл\'{а}вающих, путеш\'{е}ствующих, нед\'{у}гующих, стр\'{а}ждущих, плен\'{е}нных и о спас\'{е}нии их, Г\'{о}споду пом\'{о}\-лимся.}{}{For travellers by land, by sea, and by air; for the sick and the suffering; for captives and their salvation, let us pray to the Lord.}
\ChoirSays{Г\'{о}споди, пом\'{и}луй.}{Gospodi, pomiluj.}{Lord, have mercy.}
\DeaconSays{О изб\'{а}витися нам от вс\'{я}кия ск\'{о}рби, гн\'{е}ва и н\'{у}жды, Г\'{о}споду пом\'{о}лимся.}{}{For our deliverance from all affliction, wrath, danger, and necessity, let us pray to the Lord.}
\ChoirSays{Г\'{о}споди, пом\'{и}луй.}{Gospodi, pomiluj.}{Lord, have mercy.}
\DeaconSays{Заступ\'{и}, спас\'{и}, пом\'{и}луй и сохран\'{и} нас, Б\'{о}же, Тво\'{е}ю благод\'{а}тию.}{}{Help us, save us, have mercy on us, and keep us, O God, by Thy grace.}
\ChoirSays{Г\'{о}споди, пом\'{и}луй.}{Gospodi, pomiluj.}{Lord, have mercy.}
\DeaconSays{Пресвят\'{у}ю, преч\'{и}стую, преблагослов\'{е}нную, сл\'{а}в\-ную Влад\'{ы}чицу н\'{а}шу Богор\'{о}дицу и Приснод\'{е}ву Мар\'{и}ю со вс\'{е}ми свят\'{ы}ми помян\'{у}вше, с\'{а}ми себ\'{е} и друг др\'{у}га, и весь жив\'{о}т наш Христ\'{у} Б\'{о}гу предад\'{и}м.}{}{Commemorating our most holy, most pure, most blessed and glorious Lady Theotokos and Ever-Virgin Mary with all the saints, let us commend ourselves and each other and all our life unto Christ our God.}
\ChoirSays{Теб\'{е}, Г\'{о}споди.}{Tebje, Gospodi}{To Thee, O Lord.}

\begin{comment}
Молитва перваго антифона.
Иерей: Г\'{о}споди Б\'{о}же наш, Ег\'{о}же держ\'{а}ва несказ\'{а}нна и сл\'{а}ва непостиж\'{и}ма, Ег\'{о}же м\'{и}лость безм\'{е}рна и человекол\'{ю}бие неизреч\'{е}нно. Сам, Влад\'{ы}ко, по благоутр\'{о}бию Твоем\'{у} пр\'{и}зри на ны и на свят\'{ы}й храм сей, и сотвор\'{и} с н\'{а}ми и мол\'{я}щимися с н\'{а}ми, богатыя м\'{и}лости Тво\'{я} и щедр\'{о}ты Тво\'{я}.
\end{comment}

\PriestSays{\'{Я}ко подоб\'{а}ет Теб\'{е} вс\'{я}кая сл\'{а}ва, честь и покло\-н\'{е}ние, Отц\'{у} и С\'{ы}ну и Свят\'{о}му Д\'{у}ху, н\'{ы}не и пр\'{и}сно и во в\'{е}ки век\'{о}в.}{}{For unto Thee are due all glory, honour, and worship: to the Father, and to the Son, and to the Holy Spirit, now and ever and unto ages of ages.}
\ChoirSays{Ам\'{и}нь.}{Amin.}{Amen.}

\begin{comment}
И поется первый антифон от певцев.
Антифоны изобразительные.
Первый антифон, псалом 102.
1. Благослов\'{и}, душ\'{е} мо\'{я}, Г\'{о}спода. / Благослов\'{е}н ес\'{и}, Г\'{о}споди. / Благослов\'{и}, душ\'{е} мо\'{я}, Г\'{о}спода, / и вся вн\'{у}тренняя мо\'{я} И́мя свято\'{е} Ег\'{о}.
2. Благослов\'{и}, душ\'{е} мо\'{я}, Г\'{о}спода, / и не забыв\'{а}й всех возда\'{я}ний Ег\'{о}.
1. Очищ\'{а}ющаго вся беззак\'{о}ния тво\'{я}, / исцел\'{я}ющаго вся нед\'{у}ги тво\'{я}.
2. Избавл\'{я}ющаго от истл\'{е}ния жив\'{о}т твой, / венч\'{а}ющаго тя м\'{и}лостию и щедр\'{о}тами.
1. Исполн\'{я}ющаго во благ\'{и}х жел\'{а}ние тво\'{е}: / обнов\'{и}тся, \'{я}ко \'{о}рля, \'{ю}ность тво\'{я}.
2. Твор\'{я}й м\'{и}лостыни Госп\'{о}дь, / и судьб\'{у} всем об\'{и}димым.
1. Сказ\'{а} пут\'{и} Сво\'{я} Моис\'{е}ови, / сынов\'{о}м Изр\'{а}илевым хот\'{е}ния Сво\'{я}.
2. Щедр и м\'{и}лостив Госп\'{о}дь, / долготерпел\'{и}в и многом\'{и}лостив.
1. Не до конц\'{а} прогн\'{е}вается, / ниж\'{е} в век вражд\'{у}ет.
2. Не по беззак\'{о}нием н\'{а}шим сотвор\'{и}л есть нам, / ниж\'{е} по грех\'{о}м н\'{а}шим возд\'{а}л есть нам.
1. Я́ко по высот\'{е} неб\'{е}сней от земл\'{и}, / утверд\'{и}л есть Госп\'{о}дь м\'{и}лость Сво\'{ю} на бо\'{я}щихся Ег\'{о}.
2. Ел\'{и}ко отсто\'{я}т вост\'{о}цы от з\'{а}пад, / уд\'{а}лил есть от нас беззак\'{о}ния н\'{а}ша.
1. Я́коже щ\'{е}дрит от\'{е}ц с\'{ы}ны, / ущ\'{е}дри Госп\'{о}дь бо\'{я}щихся Ег\'{о}.
2. Я́ко Той позн\'{а} созд\'{а}ние н\'{а}ше, / помян\'{у}, \'{я}ко персть есм\'{ы}.
1. Челов\'{е}к, \'{я}ко трав\'{а} дн\'{и}е ег\'{о}, / \'{я}ко цвет с\'{е}льный, т\'{а}ко оцвет\'{е}т.
2. Я́ко дух пр\'{о}йде в нем, / и не б\'{у}дет, / и не позн\'{а}ет ктом\'{у} м\'{е}ста своег\'{о}.
1. М\'{и}лость же Госп\'{о}дня от в\'{е}ка и до в\'{е}ка / на бо\'{я}щихся Ег\'{о}.
2. И пр\'{а}вда Ег\'{о} на сын\'{е}х сын\'{о}в, / хран\'{я}щих зав\'{е}т Ег\'{о}, / и п\'{о}мнящих з\'{а}поведи Ег\'{о} / твор\'{и}ти \'{я}.
1. Госп\'{о}дь на небес\'{и} угот\'{о}ва Прест\'{о}л Свой, / и Ц\'{а}рство Ег\'{о} вс\'{е}ми облад\'{а}ет.
2. Благослов\'{и}те Г\'{о}спода, вси А́нгели Ег\'{о}, / с\'{и}льнии кр\'{е}постию, твор\'{я}щии сл\'{о}во Ег\'{о}, / усл\'{ы}шати глас слов\'{е}с Ег\'{о}.
1. Благослов\'{и}те Г\'{о}спода, вся с\'{и}лы Ег\'{о}, / слуг\'{и} Ег\'{о}, твор\'{я}щии в\'{о}лю Ег\'{о}.
2. Благослов\'{и}те Г\'{о}спода, вся дел\'{а} Ег\'{о}, / на вс\'{я}ком м\'{е}сте влад\'{ы}чества Ег\'{о}.
1. Сл\'{а}ва Отц\'{у} и С\'{ы}ну и Свят\'{о}му Д\'{у}ху.
2. И н\'{ы}не и пр\'{и}сно и во в\'{е}ки век\'{о}в. Ам\'{и}нь.
1. Благослов\'{и}, душ\'{е} мо\'{я}, Г\'{о}спода, / и вся вн\'{у}тренняя мо\'{я} И́мя свято\'{е} Ег\'{о}. / Благослов\'{е}н ес\'{и}, Г\'{о}споди.
\end{comment}

\ifIncludeMusic
\HymnFullPage{First_Antiphon}
\else
\begin{paracol}{2}
\LITTITLE{Први Антифон}
\switchcolumn
\LITTITLE{First Antiphon}
\ChoirSings{Слава Отцу и Сыну и Святому Духу, и ныне и присно и во веки веков. Ам\'{и}нь.}{}{Glory to the Father, and to the Son, and to the Holy Spirit, both now and ever, and unto ages of ages. Amen.}
\IbidSings{Благослови, душе моя, Господа, и вся внутренняя моя имя святое Его. Благословен еси Г\'{о}споди.}{Blagoslovi, dušje moja, Gospoda, i vsja vnutrjennjaja moja imja svjatoje Jego. Blagoslovjen jesi Gospodi.}{Bless the Lord, O my soul, and all that is within me, bless His holy Name. Blessed art Thou, O Lord.}
\end{paracol}
\fi


\LiturgyHeader{Малая ектения}{}{Small Litany}
%\subsection{Small Litany}
\begin{comment}
Молитва втораго антифона.
Иерей: Г\'{о}споди Б\'{о}же наш, спас\'{и} л\'{ю}ди Тво\'{я} и благослов\'{и} досто\'{я}ние Тво\'{е}, исполн\'{е}ние Ц\'{е}ркве Твое\'{я} сохран\'{и}, освят\'{и} л\'{ю}бящия благол\'{е}пие д\'{о}му Твоег\'{о}. Ты т\'{е}х воспросл\'{а}ви Бож\'{е}ственною Тво\'{е}ю с\'{и}лою и не ост\'{а}ви нас, упов\'{а}ющих на Тя.
\end{comment}

\DeaconSays{П\'{а}ки и п\'{а}ки, м\'{и}ром Г\'{о}споду пом\'{о}лимся.}{}{Again and again in peace, let us pray to the Lord.}
\ChoirSays{Г\'{о}споди, пом\'{и}луй.}{Gospodi, pomiluj.}{Lord, have mercy.}
\DeaconSays{Заступ\'{и}, спас\'{и}, пом\'{и}луй и сохран\'{и} нас, Б\'{о}же, Тво\'{е}ю благод\'{а}тию.}{}{Help us, save us, have mercy on us, and keep us, O God, by Thy grace.}
\ChoirSays{Г\'{о}споди, пом\'{и}луй.}{Gospodi, pomiluj.}{Lord, have mercy.}
\DeaconSays{Пресвят\'{у}ю, преч\'{и}стую, преблагослов\'{е}нную, сл\'{а}в\-ную Влад\'{ы}чицу н\'{а}шу Богор\'{о}дицу и Приснод\'{е}ву Мар\'{и}ю со вс\'{е}ми свят\'{ы}ми помян\'{у}вше, с\'{а}ми себ\'{е} и друг др\'{у}га, и весь жив\'{о}т наш Христ\'{у} Б\'{о}гу предад\'{и}м.}{}{Commemorating our most holy, most pure, most blessed and glorious Lady Theotokos and Ever-Virgin Mary with all the saints, let us commend ourselves and each other and all our life unto Christ our God.}
\ChoirSays{Теб\'{е}, Г\'{о}споди.}{Tebje, Gospodi}{To Thee, O Lord.}
%TODO: Leonidas' sheet says Возглашение speaks here
\PriestSays{\'{Я}ко Тво\'{я} держ\'{а}ва и Тво\'{е} есть Ц\'{а}рство и с\'{и}ла, и сл\'{а}ва, Отц\'{а} и С\'{ы}на и Свят\'{а}го Д\'{у}ха, н\'{ы}не и пр\'{и}сно и во в\'{е}ки век\'{о}в.}{}{For Thine is the dominion, and Thine is the Kingdom and the power and the glory; of the Father, and of the Son, and of the Holy Spirit, now and ever and unto ages of ages.}
\ChoirSays{Ам\'{и}нь.}{Amin.}{Amen.}

%TODO: fill this space?

%\subsection{Second Antiphon}

\begin{comment}
Вторый антифон, псалом 145.
1. Хвал\'{и}, душ\'{е} мо\'{я}, Г\'{о}спода. / Восхвал\'{ю} Г\'{о}спода в живот\'{е} мо\'{е}м, / по\'{ю} Б\'{о}гу моем\'{у}, д\'{о}ндеже есмь.
2. Не над\'{е}йтеся на кн\'{я}зи, на с\'{ы}ны челов\'{е}ческия, / в н\'{и}хже несть спас\'{е}ния.
1. Из\'{ы}дет дух ег\'{о}, / и возврат\'{и}тся в з\'{е}млю сво\'{ю}: / в той день пог\'{и}бнут вся помышл\'{е}ния ег\'{о}.
2. Блаж\'{е}н, ем\'{у}же Бог И\'{а}ковль пом\'{о}щник ег\'{о}, / упов\'{а}ние ег\'{о} на Г\'{о}спода Б\'{о}га своег\'{о}.
1. Сотв\'{о}ршаго н\'{е}бо и з\'{е}млю, / м\'{о}ре и вся, \'{я}же в них.
2. Хран\'{я}щаго \'{и}стину в век, / твор\'{я}щаго суд об\'{и}димым, / да\'{ю}щаго п\'{и}щу \'{а}лчущим.
1. Госп\'{о}дь реш\'{и}т оков\'{а}нныя, / Госп\'{о}дь умудр\'{я}ет слепц\'{ы}.
2. Госп\'{о}дь возв\'{о}дит низв\'{е}рженныя, / Госп\'{о}дь л\'{ю}бит пр\'{а}ведники.
1. Госп\'{о}дь хран\'{и}т приш\'{е}льцы, / с\'{и}ра и вдов\'{у} при\'{и}мет, / и путь гр\'{е}шных погуб\'{и}т.
2. Воцар\'{и}тся Госп\'{о}дь во век, / Бог твой, Си\'{о}не, в род и род.
1. Сл\'{а}ва Отц\'{у} и С\'{ы}ну и Свят\'{о}му Д\'{у}ху.
2. И н\'{ы}не и пр\'{и}сно и во в\'{е}ки век\'{о}в. Ам\'{и}нь.

Песнь Господу Иисусу Христу.
Лик: Единор\'{о}дный С\'{ы}не и Сл\'{о}ве Б\'{о}жий, Безсм\'{е}ртен Сый / и изв\'{о}ливый спас\'{е}ния н\'{а}шего р\'{а}ди / воплот\'{и}тися от Свят\'{ы}я Богор\'{о}дицы и Приснод\'{е}вы Мар\'{и}и, / непрел\'{о}жно вочелов\'{е}чивыйся, / распн\'{ы}йся же, Христ\'{е} Б\'{о}же, см\'{е}ртию смерть попр\'{а}вый, / Ед\'{и}н Сый Свят\'{ы}я Тр\'{о}ицы, / спрославл\'{я}емый Отц\'{у} и Свят\'{о}му Д\'{у}ху, спас\'{и} нас.

\end{comment}

\ifIncludeMusic
\HymnFullPage{Second_Antiphon}
\else
\begin{paracol}{2}
\LITTITLE{Други Антифон}\\
\switchcolumn
\LITTITLE{Second Antiphon}\\
\end{p\ChoirSings{
Слава Отцу и Сыну и Святому Духу
и ныне и присно и во веки веков. Ам\'{и}нь.
Единородный Сыне и Слове Божий, Безсмертен сый,
и изволивый спасения нашего ради
воплотитися от Святыя Богородицы
и Приснодевы Марии,
непреложно вочеловечивыйся;
распныйся же, Христе Боже,
смертию смерть поправый,
Един Сый Святыя Троицы,
спрославляемый
Отцу и Святому Духу,
спаси нас.
}{
Slava Otcu i Synu i Svjatomu Duhu,
i nynje i prisno i vo vjeki vjekov. Amin.
Jedinorodnyj Synje i Slovje Božij, Bjezsmjertjen syj,
i izvolivyj spasjenja našjego radi
voplotitisja ot Svjatyja Bogorodicy
i Prisnodjevy Marii,
njeprjeložno vočjelovječivyjsja;
raspnyjsja žje, Hristje Božje,
smjertiju smjert popravyj,
Jedin Syj Svjatyja Troicy,
sproslavljajemyj
Otcu i Svjatomu Duhu,
spasi nas.
}{
Glory to the Father, and to the Son, and to the Holy Spirit,
now and ever and unto ages of ages. Amen.
O Only-begotten Son and immortal Word of God,
Who for our salvation didst will to be incarnate of the Holy Theotokos and Ever-Virgin Mary,
Who without change didst become man and was crucified, Who art one of the Holy Trinity,
glorified with the Father and the Holy Spirit:
O Christ our God, trampling down death by death, save us.
}aracol}
\fi

\LiturgyHeader{Малая ектения}{}{Small Litany}
%\subsection{Small Litany}

\DeaconSays{П\'{а}ки и п\'{а}ки, м\'{и}ром Г\'{о}споду пом\'{о}лимся.}{}{Again and again in peace, let us pray to the Lord.}
\ChoirSays{Г\'{о}споди, пом\'{и}луй.}{Gospodi, pomiluj.}{Lord, have mercy.}
\DeaconSays{Заступ\'{и}, спас\'{и}, пом\'{и}луй и сохран\'{и} нас, Б\'{о}же, Тво\'{е}ю благод\'{а}тию.}{}{Help us, save us, have mercy on us, and keep us, O God, by Thy grace.}
\ChoirSays{Г\'{о}споди, пом\'{и}луй.}{Gospodi, pomiluj.}{Lord, have mercy.}
\DeaconSays{Пресвят\'{у}ю, преч\'{и}стую, преблагослов\'{е}нную, сл\'{а}в\-ную Влад\'{ы}чицу н\'{а}шу Богор\'{о}дицу и Приснод\'{е}ву Мар\'{и}ю со вс\'{е}ми свят\'{ы}ми помян\'{у}вше, с\'{а}ми себ\'{е} и друг др\'{у}га, и весь жив\'{о}т наш Христ\'{у} Б\'{о}гу предад\'{и}м.}{}{Commemorating our most holy, most pure, most blessed and glorious Lady Theotokos and Ever-Virgin Mary with all the saints, let us commend ourselves and each other and all our life unto Christ our God.}
\ChoirSays{Теб\'{е}, Г\'{о}споди.}{Tebje, Gospodi}{To Thee, O Lord.}
\begin{comment}
Молитва третияго антифона.
Иерей: И́же \'{о}бщия си\'{я} и согл\'{а}сныя даров\'{а}вый нам мол\'{и}твы, \'{и}же и двем\'{а} ил\'{и} трем соглас\'{у}ющимся о \'{и}мени Тво\'{е}м прош\'{е}ния под\'{а}ти обещ\'{а}вый, Сам и н\'{ы}не раб Тво\'{и}х прош\'{е}ния к пол\'{е}зному исп\'{о}лни, пода\'{я} нам и в насто\'{я}щем в\'{е}це позн\'{а}ние Твое\'{я} \'{и}стины, и в б\'{у}дущем жив\'{о}т в\'{е}чный д\'{а}руя.
\end{comment}
%TODO: For Thou Art?
\PriestSays{\'{Я}ко благ и человекол\'{ю}бец Бог ес\'{и}, и Теб\'{е} сл\'{а}ву возсыл\'{а}ем, Отц\'{у} и С\'{ы}ну и Свят\'{о}му Д\'{у}ху, н\'{ы}не и пр\'{и}сно и во в\'{е}ки век\'{о}в.}{}{For Thou are a good God and lovest mankind, and unto Thee we ascribe glory: to the Father, and to the Son, and to the Holy Spirit, now and ever and unto ages of ages.}
\ChoirSays{Ам\'{и}нь. \LITMOD{(велико)}}{Amin. \LITMOD{(extended)}}{Amen. \LITMOD{(extended)}}

%\subsection{Third Antiphon}
\ifIncludeMusic
\HymnFullPage{Third_Antiphon}
\else
\begin{paracol}{2}
\LITTITLE{[Трећи Антифон - страна~\pageref{Third_Antiphon}]}\\
\switchcolumn
\LITTITLE{[Third Antiphon - page~\pageref{Third_Antiphon}]}\\
\end{paracol}

\ChoirSings{Во Ц\'{а}рствии Тво\'{е}м помян\'{и} нас, Г\'{о}споди, / егд\'{а} при\'{и}деши, во Ц\'{а}рствии Тво\'{е}м.}{}{In Thy Kingdom remember us, O Lord, when Thou comest into Thy Kingdom.}
\IbidSings{Блаж\'{е}ни н\'{и}щии д\'{у}хом, / \'{я}ко тех есть Ц\'{а}рство Неб\'{е}сное.}{}{Blessed are the poor in spirit, for theirs is the Kingdom of Heaven.}
\IbidSings{Блаж\'{е}ни пл\'{а}чущии, / \'{я}ко т\'{и}и ут\'{е}шатся.}{}{Blessed are those who mourn, for they shall be comforted.}
\IbidSings{Блаж\'{е}ни кр\'{о}тции, / \'{я}ко т\'{и}и насл\'{е}дят з\'{е}млю.}{}{Blessed are the meek, for they shall inherit the earth.}
\IbidSings{Блаж\'{е}ни \'{а}лчущии и ж\'{а}ждущии пр\'{а}вды, / \'{я}ко т\'{и}и нас\'{ы}тятся.}{}{Blessed are those who hunger and thirst after righteousness, for they shall be filled.}
\IbidSings{Блаж\'{е}ни м\'{и}лостивии, / \'{я}ко т\'{и}и пом\'{и}ловани б\'{у}дут.}{}{Blessed are the merciful, for they shall obtain mercy.}
\IbidSings{Блаж\'{е}ни ч\'{и}стии с\'{е}рдцем, / \'{я}ко т\'{и}и Б\'{о}га \'{у}зрят.}{}{Blessed are the pure in heart, for they shall see God.}
\IbidSings{Блаж\'{е}ни миротв\'{о}рцы, / \'{я}ко т\'{и}и с\'{ы}нове Б\'{о}жии нарек\'{у}тся.}{}{Blessed are the peacemakers, for they shall be called the sons of God.}
\IbidSings{Блаж\'{е}ни изгн\'{а}ни пр\'{а}вды р\'{а}ди, / \'{я}ко тех есть Ц\'{а}рство Неб\'{е}сное.}{}{Blessed are whose who are persecuted for righteousness' sake, for theirs is the Kingdom of Heaven.}
\IbidSings{Блаж\'{е}ни ест\'{е}, егд\'{а} пон\'{о}сят вам, / и изжен\'{у}т, и рек\'{у}т всяк зол глаг\'{о}л на вы, лж\'{у}ще Мен\'{е} р\'{а}ди.}{}{Blessed are you when men shall revile you and persecute you, and shall say all manner of evil against you falsely for my sake.}
\IbidSings{Р\'{а}дуйтеся и весел\'{и}теся, / \'{я}ко мзда в\'{а}ша мн\'{о}га на Небес\'{е}х.}{}{Rejoice and be exceedingly glad, for great is your reward in heaven.}
\IbidSings{Сл\'{а}ва Отц\'{у} и С\'{ы}ну и Свят\'{о}му Д\'{у}ху.}{}{Glory to the Father, and to the Son, and to the Holy Spirit,}
\IbidSings{И н\'{ы}не и пр\'{и}сно и во в\'{е}ки век\'{о}в. Ам\'{и}нь.}{}{both now and ever, and unto ages of ages. Amen.}
\fi

\DeaconSays{Прем\'{у}дрость, пр\'{о}сти.}{}{Wisdom. Let us attend.}

%TODO: both sides
%TODO: check all LITTITLE
\ifIncludeMusic
\begin{HymnPartPage}\lilypondfile[quote, noindent, line-width=8.000000\in]{HYMNS_DIRECTORY/Entrance_Hymn.sub.ly}\end{HymnPartPage}
%\cleardoublepage
\clearpage
\HymnFullPage{Paraskeva_Tropar}
\HymnFullPage{Paraskeva_Kondak}
\HymnFullPage{Theotokos_Kondak}
%\cleardoublepage
\clearpage
\else
\begin{paracol}{2}
\LITTITLE{}\\
\switchcolumn
\LITTITLE{Entrance Hymn}\\
\end{paracol}

\begin{paracol}{2}
%TODO: correct these lyrics
%TODO: Приид\'{и}те, поклон\'{и}мся и припад\'{е}м ко Христ\'{у}. / Спас\'{и} ны, С\'{ы}не Б\'{о}жий, / Воскрес\'{ы}й из м\'{е}ртвых, по\'{ю}щия Ти, / аллил\'{у}иа.
%TODO: Во свят\'{ы}х д\'{и}вен сый, по\'{ю}щия Ти, / аллил\'{у}иа.
Приидите поклонимся
и припадем ко Христу спаси ни Сыне Божий\\
\-\hspace{2em}[воскресый из мертвых]\\
\-\hspace{2em}во святы[воскх дивен сый]\\
\-\hspace{2em}молитвам[воски Богородицы]\\
\-\hspace{2em}поющия ти:\\
\-\hspace{2em}Аллилуйя. \LITMOD{(3x)}
\switchcolumn
Come, let us worship
and fall down before Christ \\
\-\hspace{2em}[Who rose from the dead], \\
\-\hspace{2em}[Who is wonderful in His saints], \\
\-\hspace{2em}[Through the prayers of the Theotokos], \\
\-\hspace{2em}O Son of God, save us who sing to Thee:\\
\-\hspace{2em}Alliluia. \LITMOD{(3x)}
\end{paracol}

\subsection{Troparia and Kontakia}

\LITNOTE{The following hymns are sung regularly. Variable hymns are interspersed throughout.}

\LITTITLE{[Tropar to Saint Paraskeva - page~\pageref{Paraskeva_Tropar}]}\\

\begin{paracol}{2}
{
\selectlanguage{russian}
\LITTITLE{Тропарь преподобной Параскевы Сербской}
\LITSUB{Глaс 4}

Пуст\'{ы}нное и безм\'{о}лвное жити\'{е} возлюб\'{и}вши, / \\
и во след Христ\'{а}, Жених\'{а} твоег\'{о}, ус\'{е}рдно пот\'{е}кши, / \\
и Тог\'{о} благ\'{о}е \'{и}го во \'{ю}ности твоей вз\'{е}мши, / \\
Кр\'{е}стным зн\'{а}мением к м\'{ы}сленным враг\'{о}м м\'{у}жески вооруж\'{и}вшися, / \\
п\'{о}стническими п\'{о}двиги, пост\'{о}м, и мол\'{и}твами, и сл\'{е}зными к\'{а}плями / \\
\'{у}глие страст\'{е}й угас\'{и}ла ес\'{и}, достосл\'{а}вная Параск\'{е}во: / \\
и н\'{ы}не в Н\'{е}беснем черт\'{о}зе / \\
с м\'{у}дрыми д\'{е}вами предсто\'{я}щи Христ\'{у}, / \\
мол\'{и} о нас почит\'{а}ющих честн\'{у}ю п\'{а}мять тво\'{ю}.
}
\switchcolumn

\LITTITLE{Tropar to Saint Paraskeva of Serbia}
\LITSUB{Tone 4}

You loved the silent and solitary life, \\
and fervently followed Christ your Bridegroom. \\
Having taken His easy yoke in your youth, \\
and having manfully armed yourself by the Sign of the Cross against your spiritual enemies; \\
Through ascetic labours, fasting, and prayers, and by your tears, \\
you extinguished the coals of passions, O glorious Paraskeva. \\
And now as you stand in the heavenly bride chamber \\
of the wise virgins in the presence of Christ, \\
pray for us who venerate your precious memory.

\end{paracol}

\LITTITLE{[Kondak to Saint Paraskeva - page~\pageref{Paraskeva_Kondak}]}\\

\begin{paracol}{2}

{
\selectlanguage{russian}
\LITTITLE{Кондак преподобной Параскевы Сербской}
\LITSUB{Глaс 6}

Свят\'{у}ю вси заст\'{у}пницу с\'{у}щим в бед\'{а}х,/ \\
благоч\'{е}стно воспо\'{и}м Параск\'{е}ву всечестн\'{у}ю:/ \\
та бо, жити\'{е} ост\'{а}вльши тл\'{е}нное/ \\
и нетл\'{е}емое при\'{я}т бо в\'{е}ки,/ \\
сег\'{о} р\'{а}ди сл\'{а}ву обр\'{е}те,/ \\
чуд\'{е}с благод\'{а}ть Б\'{о}жиим вел\'{е}нием.
}

\switchcolumn

\LITTITLE{Kondak to Saint Paraskeva of Serbia}
\LITSUB{Tone 4}

Let us all sing hymns to Saint Paraskeva, \\
a protector of those who are in peril. \\
She forsook the life corruptible \\
and gained the one which is incorruptible; \\
wherefore, by the will of God, she was granted glory \\
and the power to perform miracles.

\end{paracol}

\LITTITLE{[Kondak to the Theotokos - page~\pageref{Theotokos_Kondak}]}

\begin{paracol}{2}

{
\selectlanguage{russian}
\LITTITLE{Конд\'{а}к Богор\'{о}дицы}
\LITSUB{Глaс 6}

Предст\'{а}тельство христи\'{а}н непост\'{ы}дное,/ \\
ход\'{а}тайство ко Творц\'{у} непрел\'{о}жное,/ \\
не пр\'{е}зри гр\'{е}шных мол\'{е}ний гл\'{а}сы,/ \\
но предвар\'{и}, \'{я}ко Благ\'{а}я,/ \\
на п\'{о}мощь нас, в\'{е}рно зов\'{у}щих Ти;/ \\
ускор\'{и} на мол\'{и}тву и потщ\'{и}ся на умол\'{е}ние,// \\
предст\'{а}тельствующи пр\'{и}сно,/ \\
Богор\'{о}дице, чт\'{у}щих Тя.
}
\switchcolumn

\LITTITLE{Kondak to the Theotokos}
\LITSUB{Tone 6}

O protection of Christians that cannot be put to shame, / \\
you are the most constant mediation to the Creator. / \\
O despise not the suppliant voices of those who have sinned;/ \\
but be quick, O good one, / \\
to come to our aid, who in faith cry to you: / \\
Hasten to intercession, and speed to make supplication, / \\
you who always protect, / \\
O Theotokos, them that honour you.

\end{paracol}
\fi


%TODO: accents here
\DeaconSays{Господу помолимся.}{}{Let us pray to the Lord.}
\ChoirSays{Г\'{о}споди, пом\'{и}луй.}{Gospodi, pomiluj.}{Lord, have mercy.}
%TODO: and here
\PriestSays{Яко свят еси, Боже наш, и Тебе славу возсылаем, Отцу и Сыну, и Святому Духу, ныне и присно\ldots}{}{For holy art Thou, O our God, and unto Thee we ascribe glory: to the Father, and to the Son, and to the Holy Spirit, now and ever\ldots}
\DeaconSays{\ldots и во веки веков.}{}{\ldots and unto ages of ages.}
\ChoirSays{Ам\'{и}нь.}{Amin.}{Amen.}

\ifIncludeMusic
%\LITTITLE{[Thrice Holy Hymn- page~\pageref{Thrice_Holy_Hymn}]}
\HymnFullPage{Thrice_Holy_Hymn}
\else
%TODO:  Трижды means thrice.
\ChoirSings{Свят\'{ы}й Б\'{о}же, Свят\'{ы}й Кр\'{е}пкий, Свят\'{ы}й Безсм\'{е}ртный, пом\'{и}луй нас. \LITMOD{(3x)}}{}{Holy God, Holy Mighty, Holy Immortal, Have mercy on us. \LITMOD{(3x)}}
\IbidSings{Сл\'{а}ва Отц\'{у} и С\'{ы}ну и Свят\'{о}му Д\'{у}ху, и н\'{ы}не и пр\'{и}сно и во в\'{е}ки век\'{о}в. Ам\'{и}нь.}{}{Glory to the Father, and to the Son, and to the Holy Spirit, both now and ever, and unto ages of ages. Amen.}
\IbidSings{Свят\'{ы}й Безсм\'{е}ртный, пом\'{и}луй нас.}{}{Holy Immortal, Have mercy on us.}
\IbidSings{Свят\'{ы}й Б\'{о}же, Свят\'{ы}й Кр\'{е}пкий, Свят\'{ы}й Безсм\'{е}ртный, пом\'{и}луй нас.}{}{Holy God, Holy Mighty, Holy Immortal, Have mercy on us.}
\fi

%TODO: accents
\DeaconSays{В\'{о}нмем.}{}{Let us attend.}
\PriestSays{Мир всем.}{}{Peace be unto all.}
\ChoirSays{И д\'{у}хови твоем\'{у}.}{I duhovi tvojemu.}{And to your spirit.}
\DeaconSays{Прем\'{у}дрость.}{}{Wisdom.}
%TODO: find Slavonic
\ReaderSays{}{}{The prokeimenon, from the Psalm of David\ldots}
\DeaconSays{Премудрость.}{}{Wisdom.}
%TODO: find Slavonic
\ReaderSays{}{}{\LITNOTE{(the reader says the title of the lesson)}}
\DeaconSays{Вонмем.}{}{Let us attend.}
%TODO: both languages
\LITNOTE{The Epistle is read\ldots}
\PriestSays{Мир ти.}{}{Peace be unto you.}
\ChoirSays{И д\'{у}хови твоем\'{у}.}{I duhovi tvojemu.}{And to your spirit.}
\DeaconSays{Прем\'{у}дрость.}{}{Wisdom.}
\LITNOTE{[Sing the following 1 to 3 times as a refrain]}
\ifIncludeMusic
\begin{HymnPartPage}\lilypondfile[quote, noindent, line-width=8.000000\in]{HYMNS_DIRECTORY/Alliluia.sub.ly}\end{HymnPartPage}
\else
\ChoirSings{Аллил\'{у}иа. \LITMOD{(3x)}}{Allil\'{u}ia. \LITMOD{(3x)}}{Alliluia. \LITMOD{(3x)}}
\fi

\subsection{Gospel Reading}

\PriestSays{Прем\'{у}дрость, пр\'{о}сти, усл\'{ы}шим свят\'{а}го Ев\'{а}нгелия. Мир всем.}{}{Wisdom. Let us attend. Let us listen to the Holy Gospel. Peace be unto all.}
\ChoirSays{И д\'{у}хови твоем\'{у}.}{I duhovi tvojemu.}{And to your spirit.}
\begin{HymnPartPage}\lilypondfile[quote, noindent, line-width=8.000000\in]{HYMNS_DIRECTORY/I_duhovi_tvojemu.sub.ly}\end{HymnPartPage}
%\DeaconSays{От \LITALT{(име)} святаго Евангелия чтение.}{}{The reading is from the Holy Gospel according to Saint \LITALT{(name)}.}
\DeaconSays{От \LITALT{(имярек)} свят\'{а}го Ев\'{а}нгелия чт\'{е}ние.}{}{The reading is from the Holy Gospel according to Saint \LITALT{(name)}.}
%TODO: both languages
\ifIncludeMusic
\begin{HymnPartPage}\lilypondfile[quote, noindent, line-width=8.000000\in]{HYMNS_DIRECTORY/Slava_Tebje_Gospodi.sub.ly}\end{HymnPartPage}
\else
\ChoirSings{Сл\'{а}ва Теб\'{е}, Г\'{о}споди, сл\'{а}ва Теб\'{е}.}{}{Glory to Thee, O Lord, glory to Thee.}
\fi
\PriestSays{В\'{о}нмем.}{}{Let us attend.}
\LITNOTE{And the Holy Gospel is read.}
\LITNOTE{Upon its completion, the priest blesses the deacon, saying:}
\PriestSays{Мир т\'{и}, благовеств\'{у}ющему.}{}{Peace be unto you who have proclaimed the Gospel.}
%TODO: both languages
\ifIncludeMusic
\begin{HymnPartPage}\lilypondfile[quote, noindent, line-width=8.000000\in]{HYMNS_DIRECTORY/Slava_Tebje_Gospodi.sub.ly}\end{HymnPartPage}
\else
\ChoirSings{Сл\'{а}ва Теб\'{е}, Г\'{о}споди, сл\'{а}ва Теб\'{е}.}{}{Glory to Thee, O Lord, glory to Thee.}
\fi

\subsection{Augmented Litany}

\DeaconSays{Рцем вс\'{и} от все\'{я} душ\'{и}, и от всег\'{о} помышл\'{е}ния н\'{а}шего рцем.}{}{Let us say with all our soul and with all our mind, let us say.}
\ChoirSays{Г\'{о}споди, пом\'{и}луй.}{Gospodi, pomiluj.}{Lord, have mercy.}
\DeaconSays{Г\'{о}споди Вседерж\'{и}телю, Б\'{о}же от\'{е}ц н\'{а}ших, м\'{о}лим Ти ся, усл\'{ы}ши и пом\'{и}луй.}{}{O Lord Almighty, the God of our fathers, we pray Thee, hearken and have mercy.}
\ChoirSays{Г\'{о}споди, пом\'{и}луй.}{Gospodi, pomiluj.}{Lord, have mercy.}
%TODO: Father Aleksandar says another prayer here; what is it?
%TODO: There is definitely another phrase here, but Deacon Andre doesn't say it.
\DeaconSays{Пом\'{и}луй нас, Б\'{о}же, по вел\'{и}цей м\'{и}лости Тво\'{е}й, м\'{о}лим Ти ся, усл\'{ы}ши и пом\'{и}луй.}{}{Have mercy on us, O God; according to Thy great goodness, we pray Thee, hearken and have mercy.}
\ChoirSays{Г\'{о}споди, пом\'{и}луй. \LITMOD{(3x)}}{Gospodi, pomiluj. \LITMOD{(3x)}}{Lord, have mercy. \LITMOD{(3x)}}
%TODO:
%Еще молимся о православном епископстве Церкве Российския, о господине нашем высокопреосвященнейшем Митрополите N., первоиерарсе Русския Зарубежныя Церкве, о господине нашем Преосвященнейшем Епископе N., и о всей во Христе братии нашей.
%Ещ\'{е} м\'{о}лимся о Вел\'{и}ком Господ\'{и}не и отц\'{е} н\'{а}шем, Свят\'{е}йшем Патри\'{а}рхе (имярек), и о господ\'{и}не н\'{а}шем преосвящ\'{е}ннейшем митропол\'{и}те (или архиеп\'{и}скопе, или еп\'{и}скопе) (имя рек), и о всей во Христ\'{е} бр\'{а}тии н\'{а}шей.
%Jesce molimsja o svjatjejsem Patrarsje nasem Irinej i o preosvjacenjejsem Episkopje nasem Longin, i vsej vo Hristje bratiji nasej.
\DeaconSays{}{}{Again we pray for our Most Holy Patriarch Irinej, for our Most Reverend Bishop Longin, and for all our brethren in Christ.}
\ChoirSays{Г\'{о}споди, пом\'{и}луй. \LITMOD{(3x)}}{Gospodi, pomiluj. \LITMOD{(3x)}}{Lord, have mercy. \LITMOD{(3x)}}
\DeaconSays{Ещ\'{е} м\'{о}лимся о Богохран\'{и}мей стран\'{е} н\'{а}шей, вла\-ст\'{е}х и в\'{о}инстве е\'{я}, да т\'{и}хое и безм\'{о}лвное жити\'{е} пожив\'{е}м во вс\'{я}ком благоч\'{е}стии и чистот\'{е}.}{}{}
\ChoirSays{Г\'{о}споди, пом\'{и}луй. \LITMOD{(3x)}}{Gospodi, pomiluj. \LITMOD{(3x)}}{Lord, have mercy. \LITMOD{(3x)}}
%TODO: find Slavonic
%Jesce molimsja o blagovjernom i hristoljubivom strazduscem pravoslavnom rodje nasem serbstjem, i o spasenii jego.
\DeaconSays{}{}{Again we pray for our Orthodox, Christ-loving and suffering Serbian people and for their salvation.}
\ChoirSays{Г\'{о}споди, пом\'{и}луй. \LITMOD{(3x)}}{Gospodi, pomiluj. \LITMOD{(3x)}}{Lord, have mercy. \LITMOD{(3x)}}
\DeaconSays{Ещ\'{е} м\'{о}лимся о бр\'{а}тиях н\'{а}ших, свящ\'{е}нницех, священномон\'{а}сех, и всем во Христ\'{е} бр\'{а}тстве н\'{а}шем.}{}{Again we pray for our brethren, priests, ordained monks, and for all our brotherhood in Christ.}
\ChoirSays{Г\'{о}споди, пом\'{и}луй. \LITMOD{(3x)}}{Gospodi, pomiluj. \LITMOD{(3x)}}{Lord, have mercy. \LITMOD{(3x)}}
%\DeaconSays{Еще молимся о блаженных и приснопамятных святейших патриарсех православных, и благочестивых царех и благоверных царицах, и создателех святого храма сего [if it be a monastery, святыя обители сея], и о всех прежде почивших отцех и братиях наших, зде лежащих и повсюду православных.}{}{Again we pray for the blessed and ever-memorable holy Orthodox patriarchs; for pious kings and right-believing queens; and for the blessed and ever-memorable founders of this holy house; and for all our fathers and brethren, the Orthodox, departed this life before us, who here and in all the world lie asleep in the Lord.}
\DeaconSays{Ещ\'{е} м\'{о}лимся о блаж\'{е}нных и присноп\'{а}мятных свят\'{е}йших патри\'{а}рсех правосл\'{а}вных и созд\'{а}телех свят\'{а}го хр\'{а}ма сег\'{о} \LITALT{(аще во обители: свят\'{ы}я об\'{и}тели се\'{я})}, и о всех пр\'{е}жде поч\'{и}вших отц\'{е}х и бр\'{а}тиях, зде леж\'{а}щих и повс\'{ю}ду, правосл\'{а}вных.}{}{Again we pray for the blessed and ever-memorable holy Orthodox patriarchs; for pious kings and right-believing queens; and for the blessed and ever-memorable founders of this holy house; and for all our fathers and brethren, the Orthodox, departed this life before us, who here and in all the world lie asleep in the Lord.}
\ChoirSays{Г\'{о}споди, пом\'{и}луй. \LITMOD{(3x)}}{Gospodi, pomiluj. \LITMOD{(3x)}}{Lord, have mercy. \LITMOD{(3x)}}
%TODO: Find Slavonic
%Jesce molimsja o milosti, zizni, mirje, zdravii, spasenii, i posjecenii, proscenii i ostavljeni grjehov bratiji svjatago hrama sego.
\DeaconSays{щ\'{е} м\'{о}лимся о м\'{и}лости, ж\'{и}зни, м\'{и}ре, здр\'{а}вии, спас\'{е}нии, посещ\'{е}нии, прощ\'{е}нии и оставл\'{е}нии грех\'{о}в раб\'{о}в Б\'{о}жиих, бр\'{а}тии свят\'{а}го хр\'{а}ма сег\'{о} \LITALT{(аще во обители: свя\-т\'{ы}я об\'{и}тели се\'{я})}.}{}{Again we pray for mercy, life, peace, health, salvation, and visitation, pardon and remission of sins of the brethren of this holy house.}
\ChoirSays{Г\'{о}споди, пом\'{и}луй. \LITMOD{(3x)}}{Gospodi, pomiluj. \LITMOD{(3x)}}{Lord, have mercy. \LITMOD{(3x)}}
\DeaconSays{Ещ\'{е} м\'{о}лимся о плодонос\'{я}щих и доброд\'{е}ющих во свят\'{е}м и всечестн\'{е}м хр\'{а}ме сем, тружд\'{а}ющихся, по\'{ю}щих и предсто\'{я}щих л\'{ю}дех, ожид\'{а}ющих от Теб\'{е} вел\'{и}кия и бог\'{а}тыя м\'{и}лости.}{}{Again we pray for those who bring offerings and do good works in this holy and all-venerable house; for those who labour and those who sing; and for all the people here present, who await Thy great and rich mercy.}
\ChoirSays{Г\'{о}споди, пом\'{и}луй. \LITMOD{(3x)}}{Gospodi, pomiluj. \LITMOD{(3x)}}{Lord, have mercy. \LITMOD{(3x)}}
\PriestSays{\'{Я}ко м\'{и}лостив и человекол\'{ю}бец Бог ес\'{и}, и Теб\'{е} сл\'{а}ву возсыл\'{а}ем, Отц\'{у} и С\'{ы}ну и Свят\'{о}му Д\'{у}ху, н\'{ы}не и пр\'{и}сно и во в\'{е}ки век\'{о}в.}{}{For Thou art a merciful God, and lovest mankind, and unto Thee we ascribe glory: to the Father, and to the Son, and to the Holy Spirit, now and ever and unto ages of ages.}
\ChoirSays{Ам\'{и}нь.}{Amin.}{Amen.}

%\LiturgyComment{}{}{The Litany for the Departed may be said here; it is usually omitted on Sundays.}
%\LiturgyReference{}{}{Litany for the Departed:  page \pageref{Departed}}

%\LITNOTE{(omitted on Sundays)}
\subsection{Litany for the Departed (typically omitted on Sundays)}
\DeaconSays{Пом\'{и}луй нас, Б\'{о}же, по вел\'{и}цей м\'{и}лости Тво\'{е}й, м\'{о}лим Ти ся, усл\'{ы}ши и пом\'{и}луй.}{}{Have mercy on us, O God; according to Thy great goodness, we pray Thee, hearken and have mercy.}
\ChoirSays{Г\'{о}споди, пом\'{и}луй. \LITMOD{(3x)}}{Gospodi, pomiluj. \LITMOD{(3x)}}{Lord, have mercy. \LITMOD{(3x)}}
\DeaconSays{Ещ\'{е} м\'{о}лимся о упоко\'{е}нии душ ус\'{о}пших раб\'{о}в Б\'{о}жиих имярек, и о \'{е}же прост\'{и}тися им вс\'{я}кому прегре\-ш\'{е}нию, в\'{о}льному же и нев\'{о}льному.}{}{Again we pray for the repose of the soul(s) of the servant(s) of God \LITALT{(names)}, departed this life, and that they (he, she) may be pardoned all their (his, her) sins, both voluntary and involuntary.}
\ChoirSays{Г\'{о}споди, пом\'{и}луй. \LITMOD{(3x)}}{Gospodi, pomiluj. \LITMOD{(3x)}}{Lord, have mercy. \LITMOD{(3x)}}
\DeaconSays{\'{Я}ко да Госп\'{о}дь Бог учин\'{и}т д\'{у}ши их, ид\'{е}же пр\'{а}\-вед\-нии упоко\'{я}ются.}{}{That the Lord God will establish their \LITALT{(his, her)} soul\LITALT{(s)} where the just repose.}
\ChoirSays{Г\'{о}споди, пом\'{и}луй. \LITMOD{(3x)}}{Gospodi, pomiluj. \LITMOD{(3x)}}{Lord, have mercy. \LITMOD{(3x)}}
\DeaconSays{М\'{и}лости Б\'{о}жия, Ц\'{а}рства Неб\'{е}снаго и оставл\'{е}ния грех\'{о}в их у Христ\'{а}, безсм\'{е}ртнаго Цар\'{я} и Б\'{о}га н\'{а}шего, пр\'{о}сим.}{}{The mercies of God, the Kingdom of Heaven, and the remission of their (his, her) sins, let us ask of Christ, the immortal King and our God.}
\ChoirSays{Под\'{а}й, Г\'{о}споди.}{Podaj, Gospodi}{Grant this, O Lord.}
\DeaconSays{Г\'{о}споду пом\'{о}лимся.}{}{Let us pray to the Lord.}
\ChoirSays{Г\'{о}споди, пом\'{и}луй.}{Gospodi, pomiluj.}{Lord, have mercy.}
\PriestSays{\'{Я}ко Ты ес\'{и} воскрес\'{е}ние и жив\'{о}т и пок\'{о}й ус\'{о}пших раб Тво\'{и}х, \LITALT{(имярек)}, Христ\'{е} Б\'{о}же наш, и Теб\'{е} сл\'{а}ву возсыл\'{а}ем, со безнач\'{а}льным Тво\'{и}м Отц\'{е}м, и пресвят\'{ы}м и благ\'{и}м и животвор\'{я}щим Тво\'{и}м Д\'{у}хом, н\'{ы}не и пр\'{и}сно и во в\'{е}ки век\'{о}в.}{}{For Thou art the Resurrection and the Life, and the Repose of Thy servant(s) \LITALT{(names)}, who has (have) fallen asleep. O Christ our God, and unto Thee we ascribe glory, together with Thy Father, who is from everlasting, and Thine All-Holy, good, and life-creating Spirit, now and ever and unto ages of ages.}
\ChoirSays{Ам\'{и}нь.}{Amin.}{Amen.}


\subsection{Litany for the Catechumens}
%TODO: Ектения о оглашенных

\DeaconSays{Помол\'{и}теся, оглаш\'{е}ннии, Г\'{о}сподеви.}{}{Pray to the Lord, you catechumens.}
\ChoirSays{Г\'{о}споди, пом\'{и}луй.}{Gospodi, pomiluj.}{Lord, have mercy.}
\DeaconSays{В\'{е}рнии, о оглаш\'{е}нных пом\'{о}лимся, да Госп\'{о}дь пом\'{и}лует их.}{}{Let us, the faithful, pray for the catechumens, that the Lord may have mercy on them.}
\ChoirSays{Г\'{о}споди, пом\'{и}луй.}{Gospodi, pomiluj.}{Lord, have mercy.}
\DeaconSays{Оглас\'{и}т их сл\'{о}вом \'{и}стины.}{}{That He may teach them the word of Truth.}
\ChoirSays{Г\'{о}споди, пом\'{и}луй.}{Gospodi, pomiluj.}{Lord, have mercy.}
\DeaconSays{Откр\'{ы}ет им Ев\'{а}нгелие пр\'{а}вды.}{}{That He may reveal to them the Gospel of righteousness.}
\ChoirSays{Г\'{о}споди, пом\'{и}луй.}{Gospodi, pomiluj.}{Lord, have mercy.}
\DeaconSays{Соедин\'{и}т их свят\'{е}й Сво\'{е}й соб\'{о}рней и ап\'{о}столь\-стей Ц\'{е}ркви.}{}{That He may unite them to His Holy Catholic, and Apostolic Church.}
\ChoirSays{Г\'{о}споди, пом\'{и}луй.}{Gospodi, pomiluj.}{Lord, have mercy.}
\DeaconSays{Спас\'{и}, пом\'{и}луй, заступ\'{и} и сохран\'{и} их, Б\'{о}же, Тво\'{е}ю благод\'{а}тию.}{}{Help them, save them, have mercy on them, and keep them, O God, by Thy grace.}
\ChoirSays{Г\'{о}споди, пом\'{и}луй.}{Gospodi, pomiluj.}{Lord, have mercy.}
\DeaconSays{Оглаш\'{е}ннии, главы в\'{а}ша Г\'{о}сподеви приклон\'{и}те.}{}{Bow your heads unto the Lord, you catechumens.}
\ChoirSays{Тебе, Г\'{о}споди. \LITMOD{(велико*)}}{Tebje, Gospodi. \LITMOD{(extended*)}}{To Thee, O Lord. \LITMOD{(extended*)}}
\PriestSays{Да и т\'{и}и с н\'{а}ми сл\'{а}вят пречестн\'{о}е и великол\'{е}пое \'{и}мя Тво\'{е}, Отц\'{а} и С\'{ы}на и Свят\'{а}го Д\'{у}ха, н\'{ы}не и пр\'{и}сно и во в\'{е}ки век\'{о}в.}{}{That with us they may glorify Thine all-honourable and majestic name: of the Father, and of the Son, and of the Holy Spirit, now and ever and unto ages of ages.}
\ChoirSays{Ам\'{и}нь.}{Amin.}{Amen.}

\subsection{Litany of the Faithful}
%TODO: Литургия Верных

\DeaconSays{Ел\'{и}цы оглаш\'{е}ннии, изыд\'{и}те, оглаш\'{е}ннии изы\-д\'{и}те, ел\'{и}цы оглаш\'{е}ннии изыд\'{и}те. Да никто от оглаш\'{е}нных, ел\'{и}цы в\'{е}рнии, п\'{а}ки и п\'{а}ки м\'{и}ром Г\'{о}споду пом\'{о}лимся.}{}{All catechumens, depart. Depart, catechumens. All that are catechumens, depart. Let no catechumen remain. Let us, the faithful, again and again in peace in peace pray unto the Lord.}
\ChoirSays{Г\'{о}споди, пом\'{и}луй. \LITMOD{(велико*)}}{Gospodi, pomiluj. \LITMOD{(extended*)}}{Lord, have mercy. \LITMOD{(extended*)}}
\DeaconSays{Заступ\'{и}, спас\'{и}, пом\'{и}луй и сохран\'{и} нас, Б\'{о}же, Тво\'{е}ю благод\'{а}тию.}{}{Help us, save us, have mercy on us, and keep us, O God, by Thy grace.}
\ChoirSays{Г\'{о}споди, пом\'{и}луй.}{Gospodi, pomiluj.}{Lord, have mercy.}
\DeaconSays{Прем\'{у}дрость.}{}{Wisdom.}
\PriestSays{\'{Я}ко подоб\'{а}ет Теб\'{е} вс\'{я}кая сл\'{а}ва, честь и поклон\'{е}\-ние, Отц\'{у} и С\'{ы}ну и Свят\'{о}му Д\'{у}ху, н\'{ы}не и пр\'{и}сно и во в\'{е}ки век\'{о}в.}{}{For unto Thee are due all glory, honour, and worship: to the Father, and to the Son, and to the Holy Spirit, now and ever and unto ages of ages.}
\ChoirSays{Ам\'{и}нь.}{Amin.}{Amen.}
%\subsection{Small Litany}
\DeaconSays{П\'{а}ки и п\'{а}ки, м\'{и}ром Г\'{о}споду пом\'{о}лимся.}{}{Again and again in peace, let us pray to the Lord.}
\ChoirSays{Г\'{о}споди, пом\'{и}луй. \LITMOD{(велико*)}}{Gospodi, pomiluj. \LITMOD{(extended*)}}{Lord, have mercy. \LITMOD{(extended*)}}

%TODO: avoid a manual page break here?
\clearpage
\LiturgyComment{}{}{If the priest serves alone, the following four petitions are not said:}
\emph{\DeaconSays{О св\'{ы}шнем м\'{и}ре и спас\'{е}нии душ н\'{а}ших, Г\'{о}споду пом\'{о}лимся.}{}{For the peace of the whole world, for the welfare of the holy churches of God, and for the unity of all, let us pray to the Lord.}
\ChoirSays{Г\'{о}споди, пом\'{и}луй.}{Gospodi, pomiluj.}{Lord, have mercy.}
\DeaconSays{О м\'{и}ре всег\'{о} мира, благосто\'{я}нии свят\'{ы}х Б\'{о}ж\-иих церкв\'{е}й и соедин\'{е}нии всех, Г\'{о}споду пом\'{о}лимся.}{}{}
\ChoirSays{Г\'{о}споди, пом\'{и}луй.}{Gospodi, pomiluj.}{Lord, have mercy.}
\DeaconSays{О свят\'{е}м хр\'{а}ме сем и с в\'{е}рою, благогов\'{е}нием и стр\'{а}хом Б\'{о}жиим вход\'{я}щих в онь, Г\'{о}споду пом\'{о}лимся.}{}{For this holy house and for those who enter with faith, reverence, and the fear of God, let us pray to the Lord.}
\ChoirSays{Г\'{о}споди, пом\'{и}луй.}{Gospodi, pomiluj.}{Lord, have mercy.}
\DeaconSays{О избавитися нам от всякия скорби, гнева и нужды, Господу помолимся.}{}{For our deliverance from all affliction, wrath, danger, and necessity, let us pray to the Lord.}
\ChoirSays{Г\'{о}споди, пом\'{и}луй.}{Gospodi, pomiluj.}{Lord, have mercy.}}
\DeaconSays{Заступ\'{и}, спас\'{и}, пом\'{и}луй и сохран\'{и} нас, Б\'{о}же, Тво\'{е}ю благод\'{а}тию.}{}{Help us, save us, have mercy on us, and keep us, O God, by Thy grace.}
\ChoirSays{Г\'{о}споди, пом\'{и}луй.}{Gospodi, pomiluj.}{Lord, have mercy.}
\DeaconSays{Прем\'{у}дрость.}{}{Wisdom.}
\PriestSays{\'{Я}ко да под держ\'{а}вою Тво\'{е}ю всегд\'{а} хран\'{и}ми, Теб\'{е} сл\'{а}ву возсыл\'{а}ем, Отц\'{у} и С\'{ы}ну и Свят\'{о}му Д\'{у}ху, н\'{ы}не и пр\'{и}сно и во в\'{е}ки век\'{о}в.}{}{That guarded always by Thy might we may ascribe glory unto Thee: to the Father, and to the Son, and to the Holy Spirit, now and ever and unto ages of ages.}
\ChoirSays{Ам\'{и}нь. \LITMOD{(велико)}}{Amin. \LITMOD{(extended)}}{Amen. \LITMOD{(extended)}}

\ifIncludeMusic
\HymnFullPage{Cherubic_Hymn_1}
\else
\LiturgyHeader{}{}{Cherubic Hymn (First Part)}
%TODO: Херувимская песнь

\ChoirSings{И́же Херув\'{и}мы т\'{а}йно образ\'{у}юще}{}{Let us who mystically represent the Cherubim,}
\IbidSings{и животвор\'{я}щей Тр\'{о}ице Трисвят\'{у}ю песнь припев\'{а}юще,}{}{and who sing the Thrice-holy hymn to the life-creating Trinity,}
\IbidSings{вс\'{я}кое н\'{ы}не жит\'{е}йское отлож\'{и}м попеч\'{е}ние.}{}{now lay aside all earthly cares.}
\fi

\subsection{Great Entrance}

%TODO: find Slavonic for entire section
%TODO: what is said here?
%Blagocestivij i hristoljubivij rod hristianskij, ktitori i prilozniki svjatago hrama sego da pomjanet Gospod Bog vo carstvii svojem, ninje i prisno i vo vjeki vjekov.
\DeaconSays{}{}{The honourable and Christ-loving Christian people, the benefactors and donors of this holy house, may the Lord God remember in His Kingdom, always, now and ever and unto ages of ages.}
%Svjatjejsago Patriarha nasego Irinej i preosvjascenjejsago Episkopa Longin, svjasceniceskij i monaseskij cin i ves prict cerkovnij da pomjanet Gospod Bog vo carstvii svojem, vsegda, ninje i prisno i vo vjeki vjekov.
\PriestSays{}{}{Our Most Holy Patriarch Irinej, our Most Reverend Metropolitan Longin, priests, deacons, monks, and all the priesthood, may the Lord God remember in His Kingdom, always, now and ever and unto ages of ages.}
%Prezidenta i praviteljstvo strani seja i vse voinstvo da omjanet Gospod Bog vo carstvii svojem, vsegda, ninje i prisno i vo vjeki vjekov.
\IbidSings{}{}{The President of this country, all civil authorities and the armed forces, may the Lord God remember in His Kingdom, always, now and ever and unto ages of ages.}
%Strazduscuju Cerkov pravoslavnuju serbskuju i strazduscij rod nas serbskij da pomjanet Gospod Bog vo carstvii svojem, vsegda, ninje i prisno i vo vjeki vjekov.
\IbidSings{}{}{Our much-suffering Serbian Orthodox Church and Serbian people, may the Lord God remember in His Kingdom, always, now and ever and unto ages of ages.}
%Vas i vsjeh pravoslavnih hristian da pomjanet Gospod Bog vo carstvii svojem, vsegda, ninje i prisno i vo vjeki vjekov.
\IbidSings{}{}{All of you and all Orthodox Christians, may the Lord God remember in His Kingdom, always, now and ever and unto ages of ages.}
%TODO: Диакон глаголет: Вел\'{и}каго господ\'{и}на и отц\'{а} н\'{а}шего имярек, Свят\'{е}йшаго Патри\'{а}рха Моск\'{о}вскаго и все\'{я} Р\'{у}си, и господ\'{и}на н\'{а}шего преосвящ\'{е}ннейшаго имярек, митропол\'{и}та (или архиеп\'{и}скопа, или еп\'{и}скопа егоже есть область), да помян\'{е}т Госп\'{о}дь Бог во Ц\'{а}рствии Сво\'{е}м всегд\'{а}, н\'{ы}не и пр\'{и}сно и во в\'{е}ки век\'{о}в.
%TODO: Таже священник: Преосвящ\'{е}нныя митропол\'{и}ты, архиеп\'{и}скопы и еп\'{и}скопы, и весь свящ\'{е}ннический и мон\'{а}шеский чин, бр\'{а}тию и прих\'{о}жан свят\'{а}го хр\'{а}ма сег\'{о} (или: свят\'{ы}я об\'{и}тели се\'{я}), вас и всех правосл\'{а}вных христи\'{а}н, да помян\'{е}т Госп\'{о}дь Бог во Ц\'{а}рствии Сво\'{е}м всегд\'{а}, н\'{ы}не и пр\'{и}сно и во в\'{е}ки век\'{о}в.

%\ChoirSays{Ам\'{и}нь.}{Amin.}{Amen.}

\ifIncludeMusic
\HymnFullPage{Cherubic_Hymn_2}
\else
\LiturgyHeader{}{}{Cherubic Hymn (Second Part)}\\
\IbidSings{\'{Я}ко да Цар\'{я} всех под\'{ы}мем, \'{а}нгельскими нев\'{и}димо доринос\'{и}ма ч\'{и}нми. Аллил\'{у}иа, аллил\'{у}иа, аллил\'{у}иа.}{}{That we may receive the King of All, who comes invisibly upborne by the angelic hosts.}
\IbidSings{Аллилуйя. \LITMOD{(3x)}}{}{Alliluia. \LITMOD{(3x)}}
\fi

\subsection{Litany of Supplication}
%TODO: Ектения просительная

\DeaconSays{Исп\'{о}лним мол\'{и}тву н\'{а}шу Г\'{о}сподеви.}{}{Let us complete our prayer to the Lord.}
\ChoirSays{Г\'{о}споди, пом\'{и}луй.}{Gospodi, pomiluj.}{Lord, have mercy.}
\DeaconSays{О предлож\'{е}нных Честн\'{ы}х Дар\'{е}х, Г\'{о}споду пом\'{о}\-лимся.}{}{For the precious Gifts now offered, let us pray to the Lord.}
\ChoirSays{Г\'{о}споди, пом\'{и}луй.}{Gospodi, pomiluj.}{Lord, have mercy.}
\DeaconSays{О свят\'{е}м хр\'{а}ме сем, и с в\'{е}рою, благогов\'{е}нием и стр\'{а}хом Б\'{о}жиим вход\'{я}щих в онь, Г\'{о}споду пом\'{о}лимся.}{}{For this holy house and for those who enter with faith, reverence, and the fear of God, let us pray to the Lord.}
\ChoirSays{Г\'{о}споди, пом\'{и}луй.}{Gospodi, pomiluj.}{Lord, have mercy.}
\DeaconSays{О изб\'{а}витися нам от вс\'{я}кия ск\'{о}рби, гн\'{е}ва и н\'{у}жды, Г\'{о}споду пом\'{о}лимся.}{}{For our deliverance from all affliction, wrath, danger, and necessity, let us pray to the Lord.}
\ChoirSays{Г\'{о}споди, пом\'{и}луй. \LITMOD{(велико*)}}{Gospodi, pomiluj. \LITMOD{(extended*)}}{Lord, have mercy. \LITMOD{(extended*)}}
\DeaconSays{Заступ\'{и}, спас\'{и}, пом\'{и}луй и сохран\'{и} нас, Б\'{о}же, Тво\'{е}ю благод\'{а}тию.}{}{Help us, save us, have mercy on us, and keep us, O God, by Thy grace.}
\ChoirSays{Г\'{о}споди, пом\'{и}луй.}{Gospodi, pomiluj.}{Lord, have mercy.}
\DeaconSays{Дне всег\'{о} соверш\'{е}нна, св\'{я}та, м\'{и}рна и безгр\'{е}шна, у Г\'{о}спода пр\'{о}сим.}{}{That the whole day may be perfect, holy, peaceful, and sinless, let us ask of the Lord.}
\ChoirSays{Под\'{а}й, Г\'{о}споди.}{Podaj, Gospodi}{Grant this, O Lord.}
\DeaconSays{\'{А}нгела м\'{и}рна, в\'{е}рна наст\'{а}вника, хран\'{и}теля душ и тел\'{е}с н\'{а}ших, у Г\'{о}спода пр\'{о}сим.}{}{An angel of peace, a faithful guide, a guardian of our souls and bodies, let us ask of the Lord.}
\ChoirSays{Под\'{а}й, Г\'{о}споди.}{Podaj, Gospodi}{Grant this, O Lord.}
\DeaconSays{Прощ\'{е}ния и оставл\'{е}ния грех\'{о}в и прегреш\'{е}ний н\'{а}ших, у Г\'{о}спода пр\'{о}сим.}{}{For the pardon and remission of our sins and transgressions, let us ask of the Lord.}
\ChoirSays{Под\'{а}й, Г\'{о}споди.}{Podaj, Gospodi}{Grant this, O Lord.}
\DeaconSays{Д\'{о}брых и пол\'{е}зных душ\'{а}м нашим, и м\'{и}ра мiрови, у Г\'{о}спода пр\'{о}сим.}{}{For all things that are good and profitable for our souls, and peace for the world, let us ask of the Lord.}
\ChoirSays{Под\'{а}й, Г\'{о}споди.}{Podaj, Gospodi}{Grant this, O Lord.}
\DeaconSays{Пр\'{о}чее вр\'{е}мя живот\'{а} н\'{а}шего в м\'{и}ре и пока\'{я}нии сконч\'{а}ти, у Г\'{о}спода пр\'{о}сим.}{}{That we may complete the remaining time of our life in peace and repentance, let us ask of the Lord.}
\ChoirSays{Под\'{а}й, Г\'{о}споди.}{Podaj, Gospodi}{Grant this, O Lord.}
\DeaconSays{Христи\'{а}нския конч\'{и}ны живот\'{а} н\'{а}шего, безбол\'{е}з\-ненны, непост\'{ы}дны, м\'{и}рны и д\'{о}браго отв\'{е}та на стр\'{а}шнем суд\'{и}щи Христ\'{о}ве пр\'{о}сим.}{}{A Christian ending to our life: painless, blameless, and peaceful; and a good defence before the dread judgement seat of Christ, let us ask of the Lord.}
\ChoirSays{Под\'{а}й, Г\'{о}споди.}{Podaj, Gospodi}{Grant this, O Lord.}
\DeaconSays{Пресвят\'{у}ю, преч\'{и}стую, преблагослов\'{е}нную, сл\'{а}в\-ную Влад\'{ы}чицу н\'{а}шу Богор\'{о}дицу и Приснод\'{е}ву Мар\'{и}ю со вс\'{е}ми свят\'{ы}ми помян\'{у}вше, с\'{а}ми себ\'{е} и друг др\'{у}га, и весь жив\'{о}т наш Христ\'{у} Б\'{о}гу предад\'{и}м.}{}{Commemorating our most holy, most pure, most blessed and glorious Lady Theotokos and Ever-Virgin Mary with all the saints, let us commend ourselves and each other and all our life unto Christ our God.}
\ChoirSays{Теб\'{е}, Г\'{о}споди.}{Tebje, Gospodi}{To Thee, O Lord.}
\PriestSays{Щедр\'{о}тами Единор\'{о}днаго С\'{ы}на Твоег\'{о}, с Н\'{и}мже благослов\'{е}н ес\'{и}, со Пресвят\'{ы}м и Благ\'{и}м и Животвор\'{я}щим Тво\'{и}м Д\'{у}хом, н\'{ы}не и пр\'{и}сно и во в\'{е}ки век\'{о}в.}{}{Through the compassions of Thine only-begotten Son, with Whom Thou art blessed, together with Thine All-Holy, good, and life-creating Spirit, now and ever and unto ages of ages.}
\ChoirSays{Ам\'{и}нь.}{Amin.}{Amen.}
\PriestSays{Мир всем.}{}{Peace be unto all.}

\ifIncludeMusic
\begin{HymnPartPage}\lilypondfile[quote, noindent, line-width=8.000000\in]{HYMNS_DIRECTORY/I_duhovi_tvojemu2.sub.ly}\end{HymnPartPage}
\else
\ChoirSings{И д\'{у}хови твоем\'{у}.}{I d\'{u}hovi tvojem\'{u}.}{And to your spirit.}
\fi

\DeaconSays{Возл\'{ю}бим друг др\'{у}га, да едином\'{ы}слием испо\-в\'{е}мы.}{}{Let us love one another, that with one mind we may confess.}

\ifIncludeMusic
\begin{HymnPartPage}\lilypondfile[quote, noindent, line-width=8.000000\in]{HYMNS_DIRECTORY/Otca_i_Sina.sub.ly}\end{HymnPartPage}
\else
\ChoirSings{Отц\'{а} и С\'{ы}на и Свят\'{а}го Д\'{у}ха: Тр\'{о}ицу единос\'{у}щную, и неразд\'{е}льную.}{}{Father, Son, and Holy Spirit: the Trinity, one in essence, and undivided.}
\fi

\DeaconSays{Дв\'{е}ри, дв\'{е}ри! прем\'{у}дростию в\'{о}нмем.}{}{The doors, the doors! In wisdom let us attend.}

%TODO: mention that this is in Serbian (right?)
%TODO: check for accuracy
%\subsection{The Creed - Симбол Веры}
\clearpage
{\Large
\begin{paracol}{2}
{
\selectlanguage{russian}
\LITTITLE{Симбол Веры}
}
\switchcolumn
\LITTITLE{The Creed}
\end{paracol}
}
%В\'{е}рую во ед\'{и}наго Б\'{о}га Отц\'{а} Вседерж\'{и}теля, Творц\'{а} н\'{е}бу и земл\'{и}, в\'{и}димым же всем и нев\'{и}димым. И во ед\'{и}наго Г\'{о}спода Иис\'{у}са Христ\'{а}, С\'{ы}на Б\'{о}жия, Единор\'{о}днаго, И́же от Отц\'{а} рожд\'{е}ннаго пр\'{е}жде всех век. Св\'{е}та от Св\'{е}та, Б\'{о}га \'{и}стинна от Б\'{о}га \'{и}стинна, рожд\'{е}нна, несотвор\'{е}нна, единос\'{у}щна Отц\'{у}, И́мже вся б\'{ы}ша. Нас р\'{а}ди челов\'{е}к и н\'{а}шего р\'{а}ди спас\'{е}ния сш\'{е}дшаго с неб\'{е}с и воплот\'{и}вшагося от Д\'{у}ха Св\'{я}та и Мар\'{и}и Д\'{е}вы и вочелов\'{е}чшася. Расп\'{я}таго же за ны при Понт\'{и}йстем Пил\'{а}те, и страд\'{а}вша, и погреб\'{е}нна. И воскр\'{е}сшаго в тр\'{е}тий день по Пис\'{а}нием. И возш\'{е}дшаго на небес\'{а}, и сед\'{я}ща одесн\'{у}ю Отц\'{а}. И п\'{а}ки гряд\'{у}щаго со сл\'{а}вою суд\'{и}ти жив\'{ы}м и м\'{е}ртвым, Ег\'{о}же Ц\'{а}рствию не б\'{у}дет конц\'{а}. И в Д\'{у}ха Свят\'{а}го, Г\'{о}спода, Животвор\'{я}щаго, И́же от Отц\'{а} исход\'{я}щаго, И́же со Отц\'{е}м и С\'{ы}ном споклан\'{я}ема и ссл\'{а}вима, глаг\'{о}лавшаго прор\'{о}ки. Во ед\'{и}ну Свят\'{у}ю, Соб\'{о}рную и Ап\'{о}стольскую Ц\'{е}рковь. Испов\'{е}дую ед\'{и}но крещ\'{е}ние во оставл\'{е}ние грех\'{о}в. Ч\'{а}ю воскрес\'{е}ния м\'{е}ртвых, и ж\'{и}зни б\'{у}дущаго в\'{е}ка. Ам\'{и}нь.
{\Large
\PeopleSay{Вјерујем у једнога Бога, Оца, Свед\-ржитеља, Творца неба и земље и свега вид\-љивог и невидљивог.}{Vjerujem u jednoga Boga, Oca, Svjed\-ržitjela, Tvorca njeba i zjemlje i svjega vid\-livog i njevidlivog.}{I believe in one God, the Father Almighty,/ Maker of heaven and earth, and of all things visible and invisible;/}
\IbidSings{И у једног Господа Исуса Христа, Сина Божијег, Јединородног, од Оца рођеног пре свих векова; Светлост од Светлости, Бога истинитог од Бога истинитог; рођеног, не створеног, једносушног са Оцем, кроз кога је све постало;}{I u jednoga Gospoda Isusa Hrista, Sina Božijeg, Jedinorodnog, od Oca rođjenog prje svih vijekova; Svjetlost od Svjetlosti, Boga istinitog od Boga istinitog; rođjenog, nje stvorjenog, jednosušnog sa Ocjem, kroz koga je svje postalo;}{And in one Lord Jesus Christ, the Son of God,/ the Only-Begotten, Begotten of the Father before all ages,/ Light of Light, True God of True God,/ Begotten, not made; of one essence with the Father, by whom all things were made;/}
\IbidSings{Који је ради нас људи и нашега спасења сишао с небеса, и оваплотио се од Духа Светога и Марије Дјеве, и постао човек;}{Koјi je radi nas ludi i radi našjega spasjenja sišao s njebjesa, i ovaplotio sje od Duha Svjetoga i Marije Djevje, i postao čovjek;}{Who for us men and for our salvation came down from heaven,/ and was incarnate of the Holy Spirit and the Virgin Mary, and became man;/}
\IbidSings{И Који је распет за нас у време Понтија Пилата, и страдао и погребен;}{I Koјi je raspjet za nas u vrjemje Pontija Pilata, i stradao i pogrjebjen;}{And was crucified for us under Pontius Pilate, and suffered, and was buried;/}
\IbidSings{И Који је васкрсао у трећи дан по Писму;}{I Koјi je vaskrsao u trjeći dan po Pismu;}{And the third day He rose again, according to the Scriptures;/}
\IbidSings{И Који се узнео на небеса и седи с десне стране Оца;}{I Koјi sje uznjeo na njebjesa i sjedi s djesnje stranje Oca;}{And ascended into heaven, and sits at the right hand of the Father;/}
\IbidSings{И Који ће опет доћи са славом, да суди живима и мртвима, Његовом царству неће бити краја.}{I Koјi ćje opjet doći sa slavom, da sudi živima i mrtvima, Njjegovom carstvu njećje biti kraja.}{And He shall come again with glory to judge both the living and the dead; Whose kingdom shall have no end./}
\IbidSings{И у Духа Светога, Господа, Животворнога, Који од Оца исходи, Који се са Оцем и Сином заједно поштује и заједно слави, Који је говорио кроз пророке.}{I u Duha Svjetoga, Gospoda, Životvornoga, Koјi od Oca ishodi, Koјi sje sa Ocjem i Sinom zajedno poštuje i Azajedno slavi, Koјi je govorio kroz prorokje.}{And in the Holy Spirit, the Lord, the Giver of Life, Who proceeds from the Father,/ Who with the Father and the Son together is worshipped and glorified; Who spoke by the Pro\-phets;/}
\IbidSings{У једну, свету, саборну и апостолску Цркву.}{U jednu, svjetu, sabornu i apostolsku Crkvu.}{In one Holy, Catholic, and Apostolic Church./}
\IbidSings{Исповедам једно крштење за опроштење гријехова.}{Ispovjedam jedno krštjenjje za oproštjenjje grijehova.}{I confess one baptism for the remission of sins./}
\IbidSings{Чекам васкрсење мртвих, и живот будућег века. Амин.}{Čjekam vaskrsjenjje mrtvih, i život budućjeg vjeka. Amin.}{I look for the resurrection of the dead,/ and the life of the age to come. Amen.}
}

\clearpage

\subsection{Canon of the Eucharist}
%Святое Возношение
\DeaconSays{Ст\'{а}нем д\'{о}бре, ст\'{а}нем со стр\'{а}хом, в\'{о}нмем, свят\'{о}е вознош\'{е}ние в м\'{и}ре принос\'{и}ти.}{}{Let us stand aright. Let us stand with fear. Let us attend, that we may offer the holy oblation in peace.}
\ifIncludeMusic
\begin{HymnPartPage}\lilypondfile[quote, noindent, line-width=8.000000\in]{HYMNS_DIRECTORY/A_mercy_of_peace.sub.ly}\end{HymnPartPage}
\else
\ChoirSings{М\'{и}лость м\'{и}ра, ж\'{е}ртву хвал\'{е}ния.}{M\'{i}lost m\'{i}ra, žj\'{e}rtvu hval\'{e}nija.}{A mercy of peace. A sacrifice of praise.}
\fi

\PriestSays{Благод\'{а}ть Г\'{о}спода н\'{а}шего Иис\'{у}са Христ\'{а}, и люб\'{ы} Б\'{о}га и Отц\'{а}, и прич\'{а}стие Свят\'{а}го Д\'{у}ха, б\'{у}ди со вс\'{е}ми в\'{а}ми.}{}{The grace of our Lord Jesus Christ, the love of God the Father, and the communion of the Holy Spirit be with all of you.}
\ifIncludeMusic
\begin{HymnPartPage}\lilypondfile[quote, noindent, line-width=8.000000\in]{HYMNS_DIRECTORY/And_with_your_spirit.sub.ly}\end{HymnPartPage}
\else
\ChoirSings{И со д\'{у}хом тво\'{и}м.}{I so duhom tvoim.}{And with your spirit.}
\fi

\PriestSays{Гор\'{е} им\'{е}им сердц\'{а}.}{}{Let us lift up our hearts.}
\ifIncludeMusic
\begin{HymnPartPage}\lilypondfile[quote, noindent, line-width=8.000000\in]{HYMNS_DIRECTORY/We_lift_them_up.sub.ly}\end{HymnPartPage}
\else
\ChoirSings{И́мамы ко Г\'{о}споду.}{}{We lift them up unto the Lord.}
\fi

\PriestSays{Благодар\'{и}м Г\'{о}спода.}{}{Let us give thanks unto the Lord.}
\ifIncludeMusic
\begin{HymnPartPage}\lilypondfile[quote, noindent, line-width=7.000000\in]{HYMNS_DIRECTORY/It_is_meet_and_right.sub.ly}\end{HymnPartPage}
\else
\ChoirSings{Дост\'{о}йно и пр\'{а}ведно есть поклан\'{я}тися Отц\'{у} и С\'{ы}ну, и Свят\'{о}му Д\'{у}ху, Тр\'{о}ице единос\'{у}щней и неразд\'{е}льней.}{}{It is meet and right to worship the Father, and the Son, and the Holy Spirit: the Trinity, one in essence and undivided.}
\fi

\PriestSays{Поб\'{е}дную песнь по\'{ю}ще, вопи\'{ю}ще, взыв\'{а}юще и глаг\'{о}люще.}{}{Singing the triumphant hymn, shouting, proclaiming and saying:}
\ifIncludeMusic
%TODO: avoid a manual page break here?
\clearpage
\begin{HymnPartPage}\lilypondfile[quote, noindent, line-width=8.000000\in]{HYMNS_DIRECTORY/Holy_Holy_Holy.sub.ly}\end{HymnPartPage}
\else
\ChoirSings{Свят, свят, свят Госп\'{о}дь Сава\'{о}ф, исп\'{о}лнь н\'{е}бо и земл\'{я} сл\'{а}вы Твое\'{я}; ос\'{а}нна в в\'{ы}шних, благослов\'{е}н Гряд\'{ы}й во \'{и}мя Госп\'{о}дне, ос\'{а}нна в в\'{ы}шних.}{}{Holy. Holy. Holy. Lord of Sabaoth. Heaven and earth are full of Thy glory. Hosanna in the highest. Blessed is He that comes in the name of the Lord. Hosanna in the highest.}
\fi

\PriestSays{Приим\'{и}те, яд\'{и}те, си\'{е} есть Т\'{е}ло Мо\'{е}, \'{е}же за вы лом\'{и}мое во оставл\'{е}ние грех\'{о}в.}{}{Take, eat. This is My Body which is broken for you, for the remission of sins.}
\ifIncludeMusic
\begin{HymnPartPage}\lilypondfile[quote, noindent, line-width=8.000000\in]{HYMNS_DIRECTORY/Amen2.sub.ly}\end{HymnPartPage}
\else
\ChoirSings{Ам\'{и}нь.}{Amin.}{Amen.}
\fi

\PriestSays{П\'{и}йте от не\'{я} вси, си\'{я} есть Кровь Мо\'{я} Н\'{о}ваго Зав\'{е}та, \'{я}же за вы и за мн\'{о}ги излив\'{а}емая, во оставл\'{е}ние грех\'{о}в.}{}{Drink of it, all of you; for this is My Blood of the New Testament, which is shed for you and for many, for the remission of sins.}
\ifIncludeMusic
\begin{HymnPartPage}\lilypondfile[quote, noindent, line-width=8.000000\in]{HYMNS_DIRECTORY/Amen3.sub.ly}\end{HymnPartPage}
\else
\ChoirSings{Ам\'{и}нь.}{Amin.}{Amen.}
\fi

\PriestSays{Тво\'{я} от Тво\'{и}х Теб\'{е} принос\'{я}ще, о всех и за вся.}{}{Thine own of Thine own we offer unto Thee, on behalf of all and for all.}
\ifIncludeMusic
\begin{HymnPartPage}\lilypondfile[quote, noindent, line-width=8.000000\in]{HYMNS_DIRECTORY/We_praise_Thee.sub.ly}\end{HymnPartPage}
\else
\ChoirSings{Теб\'{е} по\'{е}м, Теб\'{е} благослов\'{и}м, Теб\'{е} благодар\'{и}м, Г\'{о}споди, и м\'{о}лим Ти ся, Б\'{о}же наш.}{}{We praise Thee. We bless Thee. We give thanks unto Thee, O Lord. And we pray unto Thee, O our God.}
\fi

%TODO: resume progress.doc here
\PriestSays{Изрядно о пресвятей, пречистей, преблагословенней, славней Владычице нашей Богородице и Приснодеве Марии.}{}{Especially for our most holy, most pure, most blessed and glorious Lady Theotokos and Ever-Virgin Mary.}

\ifIncludeMusic
\HymnFullPage{It_Is_Truly_Meet}
\HymnFullPage{All_Creation_Rejoices}
\HymnFullPage{Zadostoinik_Pascha} % a hymn for Pascha (between Easter and Pentecost)
\else

%TODO: format this later so that it is pretty
\subsection{Hymn to the Theotokos}
%\LITTITLE{[``It is truly meet\ldots'' - page~\pageref{It_Is_Truly_Meet}]}\\
\ChoirSings{}{}{It is truly meet to bless thee, O Theotokos,
ever blessed and most blameless, and mother of our God:
More honourable than the Cherubim, and beyond compare more glorious than the Seraphim,
who without corruption gavest birth to God the Word, the Very Theotokos,
Thee do we magnify.}

\LITNOTE{(or during Great Lent)}\\
%\LITTITLE{[``All creation rejoices\ldots'' - page~\pageref{All_Creation_Rejoices}]}\\
\ChoirSings{}{}{All of creation rejoices in thee, O full of grace:
the angels in heaven and the race of men,
O sanctified temple and spiritual paradise,
the glory of virgins,
of whom God was incarnate
and became a Child, our God before the ages.
He made thy body into a throne,
and thy womb more spacious than the heavens.
All of creation rejoices in thee, O full of grace:
Glory be to thee.}
\fi


%TODO: what is said here?
%В первых, помяни, Г\'{о}споди, православное епископство Церкве Российския и Г\'{о}сподина нашего, высокопреосвященнейшаг о митрополита N, первоиерарха Русской Зарубежной Церкви и господина нашего Преосвященнейшего N.,, их же даруй святым Твоим церквам, в мире, целых, честных, здравых, долгоденствующих право правящих слово Твоея истины.
%V pervih pomjani, Gospodi, svjatjejsago Patrijarha nasego Irinje, i preosvjascenjejsago episkopa nasego Longin, ihze daruj svjatim Tvojim Cerkvam, v mirje, cjelih, cestnih, zdravih, dolgodenstvujuscih, i pravo pravjascih slovo Tvojeja istini.
\PriestSays{}{}{Among the first, remember, O Lord, His Holiness our Patriarch Irinej, His Grace, our Bishop Longin. Grant them for Thy holy churches in peace, safety, honour, health, and length of days, rightly to define the word of Thy truth.}
\ChoirSays{И всех и вся.}{I vseh i vsja.}{And all mankind.}
\PriestSays{И даждь нам единеми усты и единем сердцем славити и воспевати пречестное и великолепое имя Твое, От\'{ц}а и С\'{ы}на и Святого Духа, ныне и присно, и во веки веков.}{}{And grant that with one mouth and one heart we may praise Thine all-honourable and majestic name: of the Father, and of the Son, and of the Holy Spirit, now and ever and unto ages of ages.}
\ChoirSays{Ам\'{и}нь.}{Amin.}{Amen.}
\PriestSays{И да будут милости великого Бога и Спаса нашего Иисуса Христа со всеми вами.}{}{And the mercies of our great God and Savior Jesus Christ shall be with all of you.}
\ChoirSays{И со духом твоим.}{I so duhom tvoim.}{And with your spirit.}

\subsection{Litany Before the Lord's Prayer}

\DeaconSays{Вся святыя помянувше, паки и паки миром Господу помолимся.}{}{Having remembered all the saints, again and again in peace let us pray to the Lord.}
\ChoirSays{Г\'{о}споди, пом\'{и}луй.}{Gospodi, pomiluj.}{Lord, have mercy.}
\DeaconSays{О принесенных и освященных честных Дарех, Господу помолимся.}{}{For the precious Gifts offered and sanctified, let us pray to the Lord.}
\ChoirSays{Г\'{о}споди, пом\'{и}луй.}{Gospodi, pomiluj.}{Lord, have mercy.}
\DeaconSays{Яко да Человеколюбец Бог наш, прием я во святый и пренебесный и мысленный Свой жертвенник, в воню благоухания духовного, возниспослет нам Божественную благодать и дар Святого Духа, помолимся.}{}{That our God, who loves mankind, receiving them upon His holy, heavenly, and ideal altar as a sweet spiritual fragrance, will send down upon us in return His divine grace and the gift of the Holy Spirit, let us pray to the Lord.}
\ChoirSays{Г\'{о}споди, пом\'{и}луй.}{Gospodi, pomiluj.}{Lord, have mercy.}
\DeaconSays{О избавитися нам от всякия скорби, гнева и нужды, Господу помолимся.}{}{For our deliverance from all affliction, wrath, danger, and necessity, let us pray to the Lord.}
\ChoirSays{Г\'{о}споди, пом\'{и}луй. \LITMOD{(велико)}}{Gospodi, pomiluj. \LITMOD{(extended)}}{Lord, have mercy. \LITMOD{(extended)}}
\DeaconSays{Заступ\'{и}, спас\'{и}, пом\'{и}луй и сохран\'{и} нас, Б\'{о}же, Тво\'{е}ю благод\'{а}тию.}{}{Help us, save us, have mercy on us, and keep us, O God, by Thy grace.}
\ChoirSays{Г\'{о}споди, пом\'{и}луй.}{Gospodi, pomiluj.}{Lord, have mercy.}
\DeaconSays{Дне всего совершенна, свята, мирна и безгрешна у Господа просим.}{}{That the whole day may be perfect, holy, peaceful, and sinless, let us ask of the Lord.}
\ChoirSays{Под\'{а}й, Г\'{о}споди.}{Podaj, Gospodi}{Grant this, O Lord.}
\DeaconSays{Ангела мирна, верна наставника, хранителя душ и телес наших у Господа просим.}{}{An angel of peace, a faithful guide, a guardian of our souls and bodies, let us ask of the Lord.}
\ChoirSays{Под\'{а}й, Г\'{о}споди.}{Podaj, Gospodi}{Grant this, O Lord.}
\DeaconSays{Прощения и оставления грехов и прегрешений наших у Господа просим.}{}{For the pardon and remission of our sins and transgressions, let us ask of the Lord.}
\ChoirSays{Под\'{а}й, Г\'{о}споди.}{Podaj, Gospodi}{Grant this, O Lord.}
\DeaconSays{Добрых и полезных душам нашим и мира м i рови у Господа просим.}{}{For all things that are good and profitable for our souls, and peace for the world, let us ask of the Lord.}
\ChoirSays{Под\'{а}й, Г\'{о}споди.}{Podaj, Gospodi}{Grant this, O Lord.}
\DeaconSays{Прочее ремя ивота ашего ире окаянии кончати оспода росим.}{}{That we may complete the remaining time of our life in peace and repentance, let us ask of the Lord.}
\ChoirSays{Под\'{а}й, Г\'{о}споди.}{Podaj, Gospodi}{Grant this, O Lord.}
\DeaconSays{Христианския ончины ивота ашего езболезненны епостыдны ирны оброго твета а трашном удищи ристове росим.}{}{A Christian ending to our life: painless, blameless, and peaceful; and a good defence before the dread judgement seat of Christ, let us ask of the Lord.}
\ChoirSays{Под\'{а}й, Г\'{о}споди.}{Podaj, Gospodi}{Grant this, O Lord.}
\DeaconSays{Соединение веры и причастие Святаго Духа испросивше, сами себе и друг друга и весь живот наш Христу Богу предадим.}{}{Having asked for the unity of Faith, and the communion of the Holy Spirit, let us commend ourselves and each other, and all our life unto Christ our God.}
\ChoirSays{Теб\'{е}, Г\'{о}споди.}{Tebje, Gospodi}{To Thee, O Lord.}
\PriestSays{И сподоби нас, Владыко, со дерзновением, неосужденно смети призывати Тебе, Небесного Бога От\'{ц}а и глаголати:}{}{And make us worthy, O Master, that with boldness and without condemnation we may dare to call on Thee, the heavenly God, as Father, and to say:}

\ifIncludeMusic
\HymnFullPage{The_Prayer}
\else
\LiturgyHeader{}{}{The Lord's Prayer}
\ChoirSings{
Отче наш, Иже еси на небесех,
да святится имя Твое. Да приидет Царствие Твое.
Да будет воля Твоя яко на небеси и на земли.
Хлеб наш насущный даждь нам днесь.
И остави нам долги наша,
якоже и мы оставляем должником нашим;
и не введи нас во искушение,
но избави нас от лукаваго.
}{}{
Our Father, who art in the heavens,
hallowed be Thy name. Thy kingdom come.
Thy will be done on earth as it is in the heavens.
Give us this day our daily bread.
And forgive us our trespasses,
as we forgive those who trespass against us.
and lead us not into temptation,
but deliver us from the evil one.
}
\fi

\PriestSays{Яко Твое есть царство, и сила, и слава, От\'{ц}а и С'{ы}на, и Святаго Духа, ныне и присно, и во веки веков.}{}{For Thine is the kingdom, and the power, and the glory: of the Father, and of the Son, and of the Holy Spirit, now and ever and unto ages of ages.}
\ChoirSays{Ам\'{и}нь.}{Amin.}{Amen.}
\PriestSays{Мир всем.}{}{Peace be unto all.}
\ChoirSays{И духови твоему.}{I duhovi tvojemu.}{And to your spirit.}
\PriestSays{Главы ваша Господеви приклоните.}{}{Bow your heads unto the Lord.}
\ChoirSays{Тебе, Г\'{о}споди. \LITMOD{(велико*)}}{Tebje, Gospodi. \LITMOD{(extended*)}}{To Thee, O Lord. \LITMOD{(extended*)}}
\PriestSays{Благодатию и щедротами и человеколюбием единородного Твоего С'{ы}на, с Ним же благословен еси со пресвятым и благим и животворящим Твоим Духом, ныне и присно, и во веки веков.}{}{Through the grace and compassion and love toward mankind of Thine only-begotten Son, with whom Thou art blessed, together with Thine All-Holy, good, and life-creating Spirit, now and ever and unto ages of ages.}
\ChoirSays{Ам\'{и}нь. \LITMOD{(велико)}}{Amin. \LITMOD{(extended)}}{Amen. \LITMOD{(extended)}}
\DeaconSays{Вонмем.}{}{Let us attend.}
\PriestSays{Святая святым.}{}{The Holy Things for the holy.}
\ChoirSays{Един свят, един Господь, Иисус Христос, во славу Бога От\'{ц}а. Ам\'{и}нь.}{Jedin svjat, jedin Gospod, Iisus Hristos, vo slavu Boga Ot\'{c}a. Am\'{i}n.}{One is Holy. One is the Lord Jesus Christ, to the glory of God the Father. Amen.}

\begin{comment}
\ifIncludeMusic
\LITTITLE{[``Praise the Lord\ldots'' - page~\pageref{Hvalite}]}\\
\else
\ChoirSings{}{}{Praise the Lord from the heavens. Praise Him in the highest. Alliluia.}
\fi
\end{comment}

%TODO: maybe add these page references in later?
%\LITTITLE{[``Rejoice in the Lord\ldots'' - page~\pageref{Raduytesya_Pravednii}]}\\
%\LITTITLE{[``Under Thy Grace\ldots'' - page~\pageref{Under_Thy_Grace}]}\\
%\LITTITLE{[``Taste and see\ldots'' - page~\pageref{Vkusite_i_vidite}]}\\
%TODO: mention that this is sung on the 3rd Sunday of Lent
%\LITTITLE{[The Cross of the Lord - page~\pageref{Cross_of_the_Lord}]}

%\LITTITLE{[``Vjechnaya Pamjat'' - page~\pageref{Vjechnaya_Pamjat}]}\\
%\LITTITLE{[Excerpt from Memorial Service - page~\pageref{Memorial}]}
%TODO: finish

%TODO: what prayers are read here?

\ifIncludeMusic
%\cleardoublepage
\clearpage
%\section{Communion Hymns}
%\HymnFullPage{Hvalite}
\begin{HymnPartPage}\lilypondfile[quote, noindent, line-width=8.000000\in]{HYMNS_DIRECTORY/Hvalite.sub.ly}\end{HymnPartPage}
	\HymnFullPage{Raduytesya_Pravednii}
\HymnFullPage{Under_Thy_Grace}
\HymnFullPage{Chashu}
\HymnFullPage{In_Eternal_Memory}
\HymnFullPage{V_Pamjat_Vjechnuju}
%TODO: this hymn might take 2 pages with Latin lyrics
\HymnFullPage{Nynje_sily_nebsnyja}
%\cleardoublepage
\clearpage
\fi

%\section{Divine Liturgy}
\subsection{Holy Communion of the Faithful}

\DeaconSays{Со страхом Божиим и верою приступите.}{}{In the fear of God and with faith draw near.}
\ifIncludeMusic
\begin{HymnPartPage}\lilypondfile[quote, noindent, line-width=8.000000\in]{HYMNS_DIRECTORY/Blagosloven.sub.ly}\end{HymnPartPage}
\else
\ChoirSings{Благословен Грядый во имя Господне: Бог Господь и явися нам.}{}{Blessed is He that comes in the name of the Lord. God is the Lord and has revealed Himself to us.}
\fi


\clearpage
\subsection{Prayer Before Holy Communion}
%TODO: include title: Молитва Пре Светог Причешћа
%TODO: the priest says something here - "Let those prepared for communion... ?"
% Text from https://bogorodicinacrkva.rs/pocetna-stranica/
{\LARGE
\PeopleSay{Верујем, Господе, и испове\-дам да си Ти заиста Христос, Син Бога живога, Који си дошао у свет да грешнике спасиш, од којих сам први ја.}{Vjerujem, Gospodje, i ispovje\-dam da si Ti zaista Hristos, Sin Boga živoga, Koјi si došao u svjet da grješ\-nikje spasiš, od koјih sam prvi ja.}{I believe, O Lord, and I confess that Thou art truly the Christ, the Son of the living God, who camest into the world to save sinners, of whom I am chief.}
\IbidSings{Још вјерујем да је ово само пречисто Тело Твоје и да је ова сама пречасна Крв Твоја.}{Јoš vjerujem da je ovo samo prječisto Tjelo Tvoje i da je ova sama prječasna Krv Tvoja.}{I believe also that this is truly Thine own most pure body, and that this is truly Thine own precious blood.}
\IbidSings{Стога Ти се молим: помилуј ме и опрости ми сагрешења моја, учињена хотимице и нехотице, речју, делом, свесно и несвесно,}{Stoga Ti sje molim: pomiluј mje i o\-prosti mi sagrješjenja moja, učinjjena hotimicje i njehoticje, rječјu, djelom, svjesno i njesvjesno,}{Therefore, I pray Thee: have mercy upon me and forgive my transgressions both voluntary and involuntary, of word and of deed, committed in knowledge or in ignorance.}
\IbidSings{И удостој ме да се без осуде причестим пречистим Тајнама Твојим за опроштење грехова и за живот вечни.}{I udostoј mje da sje bjez osudje pri\-čjestim prječistim Taјnama Tvoјim za o\-proštjenjje grjehova i za život vječni.}{And make me worthy to partake without condemnation of Thy most pure Mysteries, for the remission of my sins, and unto life everlasting.}
\IbidSings{Прими ме данас, Сине Божији, за причесника Тајне Вечере Твоје, јер нећу казати Тајну непријатељима \break Твојим; нити ћу Ти дати целив као Јуда, већ као разбојник исповедам Те: сети ме се Господе у Царству своме.}{Primi mje danas, Sinje Božiјi, za prič\-jesnika Taјnje Vječjerje Tvoje, jer nje\-ću kazati Taјnu njeprijatjelima \break Tvoјim; niti ću Ti dati cjeliv kao Јuda, vjeć kao razboјnik ispovjedam Tje: \break sjeti mje sje Gospodje u Carstvu \break svomje.}{Of Thy Mystical Supper, O Son of God, accept me today as a communicant; for I will not speak of Thy Mystery to Thine enemies, neither like Judas will I give Thee a kiss; but like the thief will I confess Thee: Remember me, O Lord, in Thy Kingdom.}
\IbidSings{Да ми причешће Светим Твојим Тајнама, Господе, не буде за суд или осуду, већ на исцелење душе и тела. Амин.}{Da mi pričješćje Svjetim Tvoјim Taјnama, Gospodje, nje budje za sud ili osudu, vjeć na iscjeljenjje dušje i tjela. Amin.}{May the communion of Thy Holy Mysteries be neither to my judgement, nor to my condemnation, O Lord, but to the healing of soul and body. Amen.}


%TODO: I'm using the version of the prayer from the liturgy book, not the laminated card that is passed out
%most pure body -> pure body
%committed in knowledge -> of knowledge



}
\clearpage

%TODO: make it clear that there are 2 arrangements, and the 1st is optional
\ifIncludeMusic
\begin{HymnPartPage}\lilypondfile[quote, noindent, line-width=8.000000\in]{HYMNS_DIRECTORY/Telo_Hristovo_1.sub.ly}\end{HymnPartPage}
	\HymnFullPage{Telo_Hristovo_2}
\HymnFullPage{Vkusite_i_vidite}
\else
\ChoirSings{Тело Христово приимите, источника безсмертнаго вкусите. Аллилуйа. \LITMOD{(3x)}}{}{Receive the Body of Christ. Taste the Fountain of Immortality. Alliluia. \LITMOD{(3x)}}
\fi

\PriestSays{Спаси, Боже, люди Твоя и благослови достояние Твое!}{}{O God, save Thy people, and bless Thine inheritance.}
%TODO: include Short Resurrection Tropar here
\ifIncludeMusic
\begin{HymnPartPage}\lilypondfile[quote, noindent, line-width=8.000000\in]{HYMNS_DIRECTORY/We_have_seen_the_True_Light.sub.ly}\end{HymnPartPage}
%TODO: Note, this is chanted between Easter and Pentecost
%\clearpage	
\LITNOTE{From Pascha until Ascension, the Paschal Tropar (``Christ is risen\ldots'') is chanted instead.}
\begin{HymnPartPage}\lilypondfile[quote, noindent, line-width=8.000000\in]{HYMNS_DIRECTORY/Paschal_Tropar_Short.sub.ly}\end{HymnPartPage}
\else
\ChoirSings{Видехом свет истинный, прияхом Духа Небесного, обретохом веру истинную, нераздельней Троице поклоняемся: Та бо нас спасла есть}{}{We have seen the true Light. We have received the heavenly spirit. We have found the true Faith. Worshipping the undivided Trinity, who has saved us.}
\fi
\PriestSays{Всегда, ныне и присно, и во веки веков.}{}{Always, now and ever and unto ages of ages.}
\ifIncludeMusic
\HymnFullPage{Let_our_mouths_be_filled}
\else
\ChoirSings{Ам\'{и}нь. Да исполнятся уста наша хваления Твоего, Г\'{о}споди, яко да поем славу Твою, яко сподобил еси нас при - частитися святым Твоим, Божественным, безсмертным и животворящим Тайнам: соблюди нас во Твоей святыни, весь день поучатися правде Твоей. Аллилуйа \LITMOD{(3x)}.}{}{Amen. Let our mouths be filled with Thy praise, O Lord, that we may sing of Thy glory; for Thou hast made us worthy to partake of Thy holy, divine, immortal, and life-creating Mysteries. Keep us in Thy holiness, that all the day we may meditate upon Thy righteousness. Alliluia \LITMOD{(3x)}.}
\fi

\subsection{Litany of Thanksgiving}

\DeaconSays{Прости приемше Божественных, святых, пречистых, бессмертных, страшных Христовых Таин, достойно благодарим Господа.}{}{Having partaken of the divine, holy, most pure, immortal, heavenly, life-creating, and awesome Mysteries of Christ, let us worthily give thanks unto the Lord.}
\ChoirSays{Г\'{о}споди, пом\'{и}луй.}{Gospodi, pomiluj.}{Lord, have mercy.}
\DeaconSays{Заступ\'{и}, спас\'{и}, пом\'{и}луй и сохран\'{и} нас, Б\'{о}же, Тво\'{е}ю благод\'{а}тию.}{}{Help us, save us, have mercy on us, and keep us, O God, by Thy grace.}
\ChoirSays{Г\'{о}споди, пом\'{и}луй.}{Gospodi, pomiluj.}{Lord, have mercy.}
\DeaconSays{День весь совершен свят, мирен и безгрешен испросивше, сами себе и друг друга и весь живот наш Христу Богу предадим.}{}{Asking that the whole day may be perfect, holy, peaceful, and sinless, let us commend ourselves and each other, and all our life unto Christ our God.}
\ChoirSays{Теб\'{е}, Г\'{о}споди.}{Tebje, Gospodi}{To Thee, O Lord.}
\PriestSays{Яко Ты еси освящение наше, и Тебе славу возсылаем, Отцу и Сыну, и Святому Духу, ныне и присно, и во веки веков.}{}{For Thou art our Sanctification, and unto Thee we ascribe glory: to the Father, and to the Son, and to the Holy Spirit, now and ever and unto ages of ages.}
\ChoirSays{Ам\'{и}нь.}{Amin.}{Amen.}
%TODO: deacon or priest?
\DeaconSays{С миром изыдем.}{}{Let us depart in peace.}
\ChoirSays{О \'{и}мени Госп\'{о}дни.}{O imeni Gospodni}{In the name of the Lord.}
\DeaconSays{Господу помолимся.}{}{Let us pray to the Lord.}
\ChoirSays{Г\'{о}споди, пом\'{и}луй.}{Gospodi, pomiluj.}{Lord, have mercy.}

\subsection{Prayer Behind the Ambo}

\PriestSays{Благословляя благословящия Тя, Г\'{о}споди, и освящаяй на Тя уповающия! Спаси люди Твоя и благослови достояние Твое, исполнение Церкве Твоея сохрани, освяти любящие благолепие дому Твоего: Ты тех воспрослави Божественною Твоею силою, и не остави нас, уповающих на Тя. Мир мирови Твоему даруй, церквам Твоим, священникам, и всем людям Твоим. Яко всякое даяние благо, и всяк дар совершен свыше есть, сходяй от Тебе От\'{ц}а светов: и Тебе славу, и благодарение, и поклонение возсылаем, Отцу и Сыну, и Святому Духу ныне и присно, и во веки веков.}{}{O Lord, Who blessest those who bless Thee, and sanctifiest those who trust in Thee: Save Thy people and bless Thine inheritance.
Preserve the fullness of Thy Church. Sanctify those who love the beauty of Thy house; glorify them in return by Thy divine power, and forsake us not who put our hope in Thee.
Give peace to Thy world, and to Thy churches, and to Thy priests, to all those in civil authority, and to all Thy people.
For every good gift and every perfect gift is from above, coming down from Thee, the Father of Lights, and unto Thee we ascribe glory, thanksgiving, and worship: to the Father, and to the Son, and to the Holy Spirit, now and ever and unto ages of ages.}

\ifIncludeMusic
%TODO: avoid a manual page break here?
\clearpage
\begin{HymnPartPage}\lilypondfile[quote, noindent, line-width=8.000000\in]{HYMNS_DIRECTORY/Blessed_is_the_Name.sub.ly}\end{HymnPartPage}
%\HymnFullPage{HYMNS_DIRECTORY/Blessed_is_the_Name}
\else
\ChoirSings{}{}{Blessed be the name of the Lord, henceforth and forevermore.}
\fi

\PriestSays{Благословение Господне на вас, Того благодатию и человеколюбием, всегда, ныне и присно, и во веки веков.}{}{The blessing of the Lord be upon you through His grace and love for mankind always, now and ever and unto ages of ages.}
\ChoirSays{Ам\'{и}нь.}{Amin.}{Amen.}
\PriestSays{Слава Тебе, Христе Боже, упование наше, слава Тебе.}{}{Glory to Thee, O Christ our God and our hope, glory to Thee.}
\ChoirSays{Слава Отцу и Сыну и Святому Духу, и ныне и присно и во веки веков. Ам\'{и}нь.}{Slava Otcu i Synu i Svjatomu Duhu, i nynje i prisno i vo vjeki vjekov. Amin.}{Glory to the Father, and to the Son, and to the Holy Spirit, both now and ever, and unto ages of ages. Amen.}
\IbidSings{Г\'{о}споди, пом\'{и}луй. \LITMOD{(3x)}}{Gospodi, pomiluj. \LITMOD{(3x)}}{Lord, have mercy. \LITMOD{(3x)}}
\IbidSings{Отче, благослови.}{\'{O}tčje, blagoslov\'{i}.}{Father, bless.}

%\subsection{Dismissal}

%TODO: what does Father Alexander say here? He commemorates more saints than this.
%TODO: find Slavonic
%\LITOPT{(Voskresij iz mertvih,)} Hristos istinij Bog nas, molitvami precistija svojeja Materi, svjatih slavnih i vsehvalnih apostol, ize vo svjatih otca nasego Jovana, arhiepiskopa Konstantinja grada Zlatoustago, i svjatago Paraskeva, i svjatago \LITALT{[jegoze jest den]}, svjatih serbskih prosvjetiteljej i ucliteljej: Simeona Mirotocivago, svjatitelja Savi, Arsenija i Maksima, Valilija Ostoskago i Petra Cetinjskago, svjatih pravednih bogootec Joakima i Ani, i vsjeh svjatih, pomilujet i spaset nas, jako blag i celovjekoljubec.
\PriestSays{}{}{May \LITOPT{(He Who rose from the dead,)} Christ our true God, through the prayers of His most pure Mother; of the holy, glorious, and all-laudable Apostles; of our father among the saints, John Chrysostom, archbishop of Constantinople; of Saint Paraskeva; of Saint \LITALT{[name]} whom we celebrate today; of Saint Simeon the Myrrh-gusher, Saints Sava, Arsenius, and Maxim, Basil of Ostrog and Peter of Cetinje, of the holy and righteous ancestors of God, Joachim and Anna; and of all the saints: have mercy on us and save us, for He is good and loves mankind.}
\ChoirSays{Ам\'{и}нь.}{Amin.}{Amen.}
\PriestSays{Молитвама Светих Отаца наших, Господе Исусе Христе Сине Божији, помилуј нас грешне.}{}{Through the prayers of our holy fathers, Lord Jesus Christ our God, have mercy on us and save us.}
\ChoirSays{Ам\'{и}нь.}{Amin.}{Amen.}

\begin{comment}
%TODO: both languages
\begin{paracol}{2}
\LITNOTE{[The sermon is given here (Serbian tradition).]}
\switchcolumn
\LITNOTE{}
\end{paracol}
\end{comment}


\clearpage
{\Large
\begin{paracol}{2}
{
\selectlanguage{russian}
\LITTITLE{Псал\'{о}м 33}
}
\switchcolumn
\LITTITLE{Psalm 33}
\end{paracol}
}
\begin{comment}
\begin{paracol}{2}
% https://web.archive.org/web/20070604081036/http://www.pomog.org/
%TODO: Wherre is accent on Имя?
%TODO: Wherre is accent on Очи?
Благословл\'{ю} Г\'{о}спода на вс\'{я}кое вр\'{е}мя,/ в\'{ы}ну хвал\'{а} Ег\'{о} во уст\'{е}х мо\'{и}х.\\
О Г\'{о}споде похв\'{а}лится душ\'{а} мо\'{я},/ да усл\'{ы}шат кр\'{о}тцыи, и возвесел\'{я}тся.\\
Возвел\'{и}чите Г\'{о}спода со мн\'{о}ю,/ и вознес\'{е}м \'{И}мя Ег\'{о} вк\'{у}пе.\\
Взыск\'{а}х Г\'{о}спода, и усл\'{ы}ша мя,/ и от вс\'{е}х скорб\'{е}й мо\'{и}х изб\'{а}ви мя.\\
Приступ\'{и}те к Нем\'{у}, и просвет\'{и}теся,/ и л\'{и}ца в\'{а}ша не посты\-д\'{я}тся.\\
Сей н\'{и}щий воззв\'{а}, и Госп\'{о}дь усл\'{ы}ша и,/ и от вс\'{е}х скорб\'{е}й ег\'{о} спас\'{е} и.\\
Ополч\'{и}тся \'{а}нгел Госп\'{о}день \'{о}крест бо\'{я}щихся Ег\'{о},/ и из\-б\'{а}вит их./ Вкус\'{и}те и в\'{и}дите, \'{я}ко благ Госп\'{о}дь:\\
блаж\'{е}н муж, \'{и}же упов\'{а}ет Нань./ Б\'{о}йтеся Г\'{о}спода, вс\'{и} свят\'{и}и Ег\'{о},\\
\'{я}ко несть лиш\'{е}ния бо\'{я}щимся Ег\'{о}./ Бог\'{а}тии обнищ\'{а}ша и взалк\'{а}ша:\\
взыск\'{а}ющии же Г\'{о}спода не лиш\'{а}тся вс\'{я}каго бл\'{а}га./ Прии\-д\'{и}те, ч\'{а}да, посл\'{у}шайте мен\'{е},\\
стр\'{а}ху Госп\'{о}дню науч\'{у} вас./ Кт\'{о} есть челов\'{е}к хот\'{я}й жив\'{о}т,\\
люб\'{я}й дни в\'{и}дети бл\'{а}ги?/ Удерж\'{и} яз\'{ы}к твой от зла,\\
и устн\'{е} тво\'{и}, \'{е}же не глаг\'{о}лати льсти./ Уклон\'{и}ся от зла и сотвор\'{и} бл\'{а}го.\\
Взыщ\'{и} м\'{и}ра, и пожен\'{и} \'{и}./ \'{О}чи Госп\'{о}дни на пр\'{а}ведныя,\\
и \'{у}ши Ег\'{о} в мол\'{и}тву их./ Лиц\'{е} же Госп\'{о}дне на твор\'{я}щия зл\'{а}я,\\
\'{е}же потреб\'{и}ти от земл\'{и} п\'{а}мять их./ Воззв\'{а}ша пр\'{а}веднии, и Госп\'{о}дь усл\'{ы}ша их,\\
и от всех скорб\'{е}й их изб\'{а}ви их./ Близ Госп\'{о}дь сокру\-ш\'{е}нных с\'{е}рдцем,\\
и смир\'{е}нныя д\'{у}хом спас\'{е}т./ Мн\'{о}ги ск\'{о}рби пр\'{а}ведным,\\
и от всех их изб\'{а}вит \'{я} Госп\'{о}дь./ Хран\'{и}т Госп\'{о}дь вся к\'{о}сти их,\\
ни ед\'{и}на от них сокруш\'{и}тся./ Смерть гр\'{е}шников лют\'{а},\\
и ненав\'{и}дящии пр\'{а}веднаго прегр\'{е}шат./ Изб\'{а}вит Госп\'{о}дь д\'{у}шы раб Сво\'{и}х,\\
и не прегреш\'{а}т// вси, упов\'{а}ющии на Нег\'{о}.

% https://web.archive.org/web/20070604081036/http://www.pomog.org/
\switchcolumn
%Psalm 33: David's. When He Changed His Countenance before Abimelech, and Was Dismissed, and Went Away, 33.
\end{paracol}
\end{comment}


\PeopleSay
{Благословл\'{ю} Г\'{о}спода на вс\'{я}кое вр\'{е}мя, / в\'{ы}ну хвал\'{а} Ег\'{о} во уст\'{е}х мо\'{и}х.}
{Blagoslovl\'{ju} G\'{o}spoda na vs\'{ja}koje vr\'{je}mja, / v\'{y}nu hval\'{a} Jeg\'{o} vo ust\'{je}h mo\'{i}h.}
{I will bless the Lord at all times, / His praise shall continually be in my mouth.}
\IbidSings
{О Г\'{о}споде похв\'{а}лится душ\'{а} мо\'{я}, / да усл\'{ы}шат кр\'{о}тцыи, и возвесел\'{я}тся.}
{O G\'{o}spodje pohv\'{a}litsja duš\'{a} mo\'{ja}, / da usl\'{y}šat kr\'{o}tcyi, i vozvjesjel\'{ja}tsja.}
{In the Lord shall my soul be praised; / let the meek hear and be glad.}
\IbidSings
{Возвел\'{и}чите Г\'{о}спода со мн\'{о}ю, / и вознес\'{е}м \'{И}мя Ег\'{о} вк\'{у}пе.}
{Vozvjel\'{i}čitje G\'{o}spoda so mn\'{o}ju, / i voznjes\'{je}m \'{I}mja Jeg\'{o} vk\'{u}pje.}
{O magnify the Lord with me, / and let us exalt His name together.}
\IbidSings
{Взыск\'{а}х Г\'{о}спода, и усл\'{ы}ша мя, / и от вс\'{е}х скорб\'{е}й мо\'{и}х изб\'{а}ви мя.}
{Vzysk\'{a}h G\'{o}spoda, i usl\'{y}ša mja, / i ot vs\'{je}h skorb\'{je}j mo\'{i}h izb\'{a}vi mja.}
{I sought the Lord, and He heard me, / and delivered me from all my tribulations.}
\IbidSings
{Приступ\'{и}те к Нем\'{у}, и просвет\'{и}теся, / и л\'{и}ца в\'{а}ша не посты\-д\'{я}тся.}
{Pristup\'{i}tje k Njem\'{u}, i prosvjet\'{i}tjesja, / i l\'{i}ca v\'{a}ša nje posty\-d\'{ja}tsja.}
{Come unto Him, and be enlightened, / and your faces shall not be ashamed.}
\IbidSings
{Сей н\'{и}щий воззв\'{а}, и Госп\'{о}дь усл\'{ы}ша и, / и от вс\'{е}х скорб\'{е}й ег\'{о} спас\'{е} и.}
{Sjej n\'{i}ŝij vozzv\'{a}, i Gosp\'{o}d usl\'{y}ša i, / i ot vs\'{je}h skorb\'{je}j jeg\'{o} spas\'{je} i.}
{This poor man cried, and the Lord heard him, / and saved him out of all his tribulations.}
\IbidSings
{Ополч\'{и}тся \'{а}нгел Госп\'{о}день \'{о}крест бо\'{я}щихся Ег\'{о}, / и из\-б\'{а}вит их.}
{Opolč\'{i}tsja \'{a}ngjel Gosp\'{o}djen \'{o}krjest bo\'{ja}ŝihsja Jeg\'{o}, / i iz\-b\'{a}vit ih.}
{The angel of the Lord will encamp round about them that fear Him, / and will deliver them.}
\IbidSings
{Вкус\'{и}те и в\'{и}дите, \'{я}ко благ Госп\'{о}дь: / блаж\'{е}н муж, \'{и}же упов\'{а}ет Нань.}
{Vkus\'{i}tje i v\'{i}ditje, \'{ja}ko blag Gosp\'{o}d: / blaž\'{je}n muž, \'{i}žje upov\'{a}jet Nan.}
{O taste and see that the Lord is good; / blessed is the man that hopeth in Him.}
\IbidSings
{Б\'{о}йтеся Г\'{о}спода, вс\'{и} свят\'{и}и Ег\'{о}, / \'{я}ко несть лиш\'{е}ния бо\'{я}\-щимся Ег\'{о}.}
{B\'{o}jtjesja G\'{o}spoda, vs\'{i} svjat\'{i}i Jeg\'{o}, / \'{ja}ko njest liš\'{je}nja bo\'{ja}\-ŝimsja Jeg\'{o}.}
{O fear the Lord, all ye His saints; / for there is no want to them that fear Him.}
\IbidSings
{Бог\'{а}тии обнищ\'{а}ша и взалк\'{а}ша: / взыск\'{а}ющии же Г\'{о}спода не лиш\'{а}тся вс\'{я}каго бл\'{а}га.}
{Bog\'{a}tii obniŝ\'{a}ša i vzalk\'{a}ša: / vzysk\'{a}juŝii žje G\'{o}spoda nje liš\'{a}tsja vs\'{ja}kago bl\'{a}ga.}
{Rich men have turned poor and gone hungry; / but they that seek the Lord shall not be deprived of any good thing.}
\IbidSings
{Прии\-д\'{и}те, ч\'{а}да, посл\'{у}шайте мен\'{е}, / стр\'{а}ху Госп\'{о}дню науч\'{у} вас.}
{Prii\-d\'{i}tje, č\'{a}da, posl\'{u}šajtje mjen\'{je}, / str\'{a}hu Gosp\'{o}dnju nauč\'{u} vas.}
{Come ye children, hearken unto me; / I will teach you the fear of the Lord.}
\IbidSings
{Кт\'{о} есть челов\'{е}к хот\'{я}й жив\'{о}т, / люб\'{я}й дни в\'{и}дети бл\'{а}ги.}
{Kt\'{o} jest čjelov\'{je}k hot\'{ja}j živ\'{o}t, / ljub\'{ja}j dni v\'{i}djeti bl\'{a}gi.}
{What man is there that desireth life, / who loveth to see good days?}
\IbidSings
{Удерж\'{и} яз\'{ы}к твой от зла, / и устн\'{е} тво\'{и}, \'{е}же не глаг\'{о}лати льсти.}
{Udjerž\'{i} jaz\'{y}k tvoj ot zla, / i ustn\'{je} tvo\'{i}, \'{je}žje nje glag\'{o}lati lsti.}
{Keep thy tongue from evil, / and thy lips from speaking guile.}
\IbidSings
{Уклон\'{и}ся от зла и сотвор\'{и} бл\'{а}го. / Взыщ\'{и} м\'{и}ра, и пожен\'{и} \'{и}.}
{Uklon\'{i}sja ot zla i sotvor\'{i} bl\'{a}go. / Vzyŝ\'{i} m\'{i}ra, i požjen\'{i} \'{i}.}
{Turn away from evil, and do good; / seek peace, and pursue it.}
\IbidSings
{\'{О}чи Госп\'{о}дни на пр\'{а}ведныя, / и \'{у}ши Ег\'{о} в мол\'{и}тву их.}
{\'{O}či Gosp\'{o}dni na pr\'{a}vjednyja, / i \'{u}ši Jeg\'{o} v mol\'{i}tvu ih.}
{The eyes of the Lord are upon the righteous, / and His ears are opened unto their supplication.}
\IbidSings
{Лиц\'{е} же Госп\'{о}дне на твор\'{я}щия зл\'{а}я, / \'{е}же потреб\'{и}ти от земл\'{и} п\'{а}мять их.}
{Lic\'{je} žje Gosp\'{o}dnje na tvor\'{ja}ŝja zl\'{a}ja, / \'{je}žje potrjeb\'{i}ti ot zjeml\'{i} p\'{a}mjat ih.}
{The face of the Lord is against them that do evil, / utterly to destroy the remembrance of them from the earth.}
\IbidSings
{Воззв\'{а}ша пр\'{а}веднии, и Госп\'{о}дь усл\'{ы}ша их, / и от всех скорб\'{е}й их изб\'{а}ви их.}
{Vozzv\'{a}ša pr\'{a}vjednii, i Gosp\'{o}d usl\'{y}ša ih, / i ot vsjeh skorb\'{je}j ih izb\'{a}vi ih.}
{The righteous cried, and the Lord heard them, / and He delivered them out of all their tribulations.}
\IbidSings
{Близ Госп\'{о}дь сокру\-ш\'{е}нных с\'{е}рдцем, / и смир\'{е}нныя д\'{у}хом спас\'{е}т.}
{Bliz Gosp\'{o}d sokru\-š\'{je}nnyh s\'{je}rdcjem, / i smir\'{je}nnyja d\'{u}hom spas\'{je}t.}
{The Lord is nigh unto them that are of a contrite heart, / and He will save the humble of spirit.}
\IbidSings
{Мн\'{о}ги ск\'{о}рби пр\'{а}ведным, / и от всех их изб\'{а}вит \'{я} Госп\'{о}дь.}
{Mn\'{o}gi sk\'{o}rbi pr\'{a}vjednym, / i ot vsjeh ih izb\'{a}vit \'{ja} Gosp\'{o}d.}
{Many are the tribulations of the righteous, / and the Lord shall deliver them out of them all.}
\IbidSings
{Хран\'{и}т Госп\'{о}дь вся к\'{о}сти их, / ни ед\'{и}на от них сокруш\'{и}тся.}
{Hran\'{i}t Gosp\'{o}d vsja k\'{o}sti ih, / ni jed\'{i}na ot nih sokruš\'{i}tsja.}
{The Lord keepeth all their bones, / not one of them shall be broken.}
\IbidSings
{Смерть гр\'{е}шников лют\'{а}, / и ненав\'{и}дящии пр\'{а}веднаго пре\-гр\'{е}шат.}
{Smjert gr\'{je}šnikov ljut\'{a}, / i njenav\'{i}djaŝii pr\'{a}vjednago prje\-gr\'{je}šat.}
{The death of sinners is evil, / and they that hate the righteous shall do wrong.}
\IbidSings
{Изб\'{а}вит Госп\'{о}дь д\'{у}шы раб Сво\'{и}х, / и не прегреш\'{а}т вси, упов\'{а}ющии на Нег\'{о}}
{Izb\'{a}vit Gosp\'{o}d d\'{u}šy rab Svo\'{i}h, / i nje prjegrješ\'{a}t vsi, upo\-v\'{a}juŝii na Njeg\'{o}}
{The Lord will redeem the souls of His servants, / and none of them will do wrong that hope in Him.}











%\cleardoublepage
\clearpage
\section{Prayers of Thanksgiving}
%\section{Prayers of Thanksgiving After Holy Communion}

%TODO: this is Slavonic, right?
%TODO: the Slavonic in this section comes from the file that Leonidas sent me
%TODO: it also says that reader is "Чмец"

\PriestSays{Слава Тебе, Боже. \LITMOD{(3x)}}{}{Glory to Thee, O God. \LITMOD{(3x)}}

\LiturgyHeader{Молитва 1, Великаго Василия}{}{}
\ReaderSays{
Благодарю Тя, Г\'{о}споди Боже мой, яко не отринул мя еси грешнаго, но общника мя быти святынь Твоих сподобил еси. Благодарю Тя, яко мене недостойнаго, причаститися пречистых Твоих и небесных Даров сподобил еси. Но Владыко Человеколюбче, нас ради умерый же и воскресый, и даровавый нам страшная сия и животворящая Таинства, во благодеяние и освящение душ и телес наших, даждь быти сим и мне во исцеление души же и тела, во отгнание всякаго сопротивнаго, в просвещение очию сердца моего, в мир душевных моих сил, в веру непостыдну, в любовь нелицемерну, во исполнение премудрости, в соблюдение заповедей Твоих, в приложение Божественныя Твоея благодати, и Твоего Царствия присвоение, да во святыни Твоей теми сохраняемь, Твою благодать поминаю всегда, и не ктому себе живу, но Тебе, нашему Владыце и Благодетелю; и тако сего жития изшед о надежди живота вечнаго, в присносущный достигну покой, идеже празднующих глас непрестанный, и безконечная сладость зрящих Твоего лица доброту неизреченную. Ты бо еси истинное желание, и неизреченное веселие любящих Тя, Христе Боже наш, и Тя поет вся тварь во веки. Ам\'{и}нь.
}{}{
I thank Thee, O Lord my God, for Thou has not rejected me, a sinner, but hast made me worthy to be a partaker of Thy Holy Things.
I thank Thee, for Thou hast permitted me, the unworthy, to commune of Thy most pure and heavenly Gifts.
But, O Master who lovest mankind, who for our sake didst die and rise again, and gavest us these awesome and life-creating Mysteries for the good and sanctification of our souls and bodies; let them be for the healing of my soul and body, the repelling of every adversary, the illuminating of the eyes of my heart, the peace of my spiritual powers, a faith unashamed, a love unfeigned, the fulfilling of wisdom, the observing of Thy commandments, the receiving of Thy divine grace, and the attaining of Thy kingdom.
Preserved by them in Thy holiness, may I always remember Thy grace and live not for myself alone, but for Thee, our Master and Benefactor.
May I pass from this life in the hope of eternal life, and so attain to the everlasting rest, where the voice of those who feast is unceasing, and the gladness of those who behold the goodness of Thy countenance is unending.
For Thou art the true desire and the ineffable joy of those who love Thee, O Christ our God, and all creation sings Thy praise forever.
Amen.
}
\LiturgyHeader{Молитва 2, Великаго Василия}{}{A Prayer of Saint Basil the Great}
\ReaderSays{Владыко Христе Боже, Царю веков и Содетелю всех, благодарю Тя о всех, яже ми еси подал благих, и о причащении пречистых и животворящих Твоих Таинств. Молю убо Тя, Блаже и Человеколюбче: сохрани мя под кровом Твоим, и в сени крилу Твоею. И даруй ми чистою совестию, даже до последняго моего издыхания, достойно причащатися Святынь Твоих, во оставление грехов и в жизнь вечную. Ты бо еси Хлеб животный, источник святыни, податель благих, и Тебе славу возсылаем, со Отцем и Святым Духом, ныне и присно, и во веки веков. Ам\'{и}нь.}{}{O Master Christ our God, King of the ages, Maker of all things: I thank Thee for all the good things Thou hast given me, especially for the communion with Thy most pure and life-creating Mysteries. I pray Thee, O gracious Lover of man: preserve me under Thy protection, beneath the shadow of Thy wings. Enable me, even to my last breath, to partake worthily and with a pure conscience of Thy Holy Things, for the remission of sins and unto life eternal. For Thou are the Bread of Life, the Fountain of Holiness, the Giver of all good; to Thee we ascribe glory, with the Father and the Holy Spirit, now and ever and unto ages of ages. Amen.}
\LiturgyHeader{Молитва 3, Метафраста, по стихом}{}{}
\ReaderSays{Давый пищу мне плоть Твою волею, огнь сый и опаляяй недостойныя, да не опалиши мене, Содетелю мой; паче же пройди во уды моя, во вся составы, во утробу, в сердце. Попали терние всех моих прегрешений. Душу очисти, освяти помышления. Составы утверди с костьми вкупе. Чувств просвети простую пятерицу. Всего мя спригвозди страху Твоему. Присно покрый, соблюди же, и сохрани мя от всякаго дела и слова душетленнаго. Очисти, и омый, и украси мя: удобри, вразуми, и просвети мя. Покажи мя Твое селение единаго Духа, и не ктому селение греха. Да яко Твоего дому входом причащения, яко огня мене бежит всяк злодей, всяка страсть. Молитвенники Тебе приношу вся святыя, чиноначалия же безплотных, Предтечу Твоего, премудрыя апостолы, к сим же Твою нескверную, чистую Матерь, ихже мольбы, Благоутробне, приими, Христе мой, и сыном света соделай Твоего служителя. Ты бо еси освящение, и Единый наших, Блаже, душ и светлость, и Тебе лепоподобно яко Богу и Владыце славу вси возсылаем на всяк день.}{}{Freely Thou hast given me Thy Body for my food, O Thou who art a fire consuming the unworthy. Consume me not, O my Creator, but instead enter into my members, my veins, my heart. Consume the thorns of my many transgressions. Cleanse my soul and sanctify my reasonings. Make firm my knees and body. Illumine my five senses. Nail me to the fear of Thee. Always protect, guard, and keep me. Give me understanding and illumination. Show me to be a temple of Thy One Spirit, and not the home of many sins. May every evil thing, every carnal passion, flee from me as from a fire as I become Thy tabernacle through communion. I offer Thee as intercessors all the saints: the leaders of the bodiless hosts, Thy Forerunner, the wise apostles, and Thy pure and blameless Mother. Accept their prayers in Thy love, O my Christ, and make me, Thy servant, a child of light. For Thou are the only Sanctification and Light of our souls, O Good One, and to Thee, our Master and God, we ascribe glory day by day.}
\LiturgyHeader{Молитва 4, иная}{}{Another Prayer}
\ReaderSays{Тело Твое Святое, Г\'{о}споди Иисусе Христе Боже наш, да будет ми в живот вечный, и Кровь Твоя Честная во оставление грехов: буди же ми благодарение сие в радость, здравие и веселие; в страшное же и второе пришествие Твое сподоби мя, грешнаго, стати одесную славы Твоея, молитвами Пречистыя Твоея Матери, и всех святых.}{}{O Lord Jesus Christ our God: let Thy holy Body be my eternal life; Thy precious Blood, my remission of sins. Let this Eucharist be my joy, health, and gladness. Make me, a sinner, worthy to stand on the right hand of Thy glory and Thine awesome second coming, through the prayers of Thy most pure Mother and of all the saints.}
\LiturgyHeader{Молитва 5, иная, ко Пресвятой Богородице}{}{A Prayer to the Theotokos}
\ReaderSays{Пресвятая Владычице Богородице, свете помраченныя моея души, надеждо, покрове, прибежище, утешение, радование мое, благодарю Тя, яко сподобила мя еси недостойнаго, причастника быти Пречистаго Тела, и Честныя Крове С'{ы}на Твоего. Но рождшая истинный Свет, просвети моя умныя очи сердца; Яже источник безсмертия рождшая, оживотвори мя умерщвленнаго грехом; Яже милостиваго Бога любоблагоутробная Мати, помилуй мя, и даждь ми умиление, и сокрушение в сердце моем, и смирение в мыслех моих, и воззвание в пленениих помышлений моих; и сподоби мя до последняго издыхания, неосужденно приимати Пречистых Таин освящение во исцеление души же и тела. И подаждь ми слезы покаяния и исповедания, во еже пети и славити Тя во вся дни живота моего, [громко/гласно] яко благословенна и препрославленна еси во веки. Ам\'{и}нь.}{}{O most holy Lady Theotokos, the light of my darkened soul, my hope, my protection, my refuge, my rest, and my joy. I thank you, for you have permitted me, the unworthy, to be a partaker of the most Pure Body and Precious Blood of your Son. Give the light of understanding to the eyes of my heart, you that gave birth to the True Light. Enliven me who am deadened by sin, you that gave birth to the Fountain of Immortality. Have mercy on me, O loving Mother of the merciful God. Grant me compunction and contrition of heart, humility in my thoughts, and a release from the slavery of my own reasonings. And enable me, even to my last breath, to receive the sanctification of the most pure Mysteries, for the healing of soul and body. Grant me tears of repentance and confession, that I may glorify you all the days of my life, for you are blessed and greatly glorified forever. Amen.}
\PriestSays{Яко благословенна и препрославленна еси во веки. Ам\'{и}нь.}{}{You are blessed and greatly glorified forever. Amen.}
\ReaderSays{Яко благословенна и препрославленна еси во веки. Ам\'{и}нь.}{}{You are blessed and greatly glorified forever. Amen.}

%\subsection{Dismissal}

\PriestSays{Ныне отпущаеши раба Твоего, Владыко, / по глаголу Твоему с миром, / яко видеста очи мои спасение Твое, / еже еси уготовал пред лицем всех людей, / свет во откровение языков / и славу людей Твоих Израиля.}{Nynje otpuŝaješi raba Tvojego, Vladyko, / po glagolu Tvojemu s mirom, / jako vidjesta oči moi spasjenije Tvoje, / ježje jesi ugotoval prjed licjem vsjeh ljudjej, / svjet vo otkrovjenije jazykov / i slavu ljudjej Tvoih Izrailja.}{Lord, now lettest Thy servant depart in peace, according to Thy word. For mine eyes have seen Thy salvation which Thou hast prepared before all people: a light to lighten the Gentiles, and the glory of Thy people Israel.}
\ReaderSays{Свят\'{ы}й Б\'{о}же, Свят\'{ы}й Кр\'{е}пкий, Свят\'{ы}й Безсм\'{е}рт\-ный, пом\'{и}луй нас. \LITMOD{(3x)}}{Svjat\'{y}j B\'{o}žje, Svjat\'{y}j Kr\'{je}pkij, Svjat\'{y}j Bjezsm\'{je}rt\-nyj, pom\'{i}luj nas. \LITMOD{(3x)}}{Holy God, Holy Mighty, Holy Immortal, have mercy on us. \LITMOD{(3x)}}
\IbidSings{Сл\'{а}ва Отц\'{у} и С\'{ы}ну и Свят\'{о}му Д\'{у}ху, и н\'{ы}не и пр\'{и}сно и во в\'{е}ки век\'{о}в. Ам\'{и}нь.}{Sl\'{a}va Otc\'{u} i S\'{y}nu i Svjat\'{o}mu D\'{u}hu, i n\'{y}nje i pr\'{i}sno i vo v\'{je}ki vjek\'{o}v. Am\'{i}n.}{Glory to the Father, and to the Son, and to the Holy Spirit, now and ever and unto ages of ages. Amen.}
\IbidSings{Пресвят\'{а}я Тр\'{о}ице, пом\'{и}луй нас; Г\'{о}споди, оч\'{и}сти грех\'{и} н\'{а}ша; Влад\'{ы}ко, прост\'{и} беззак\'{о}ния н\'{а}ша; Свят\'{ы}й, посет\'{и} и исцел\'{и} н\'{е}мощи н\'{а}ша, \'{и}мене Тво\'{е}го р\'{а}ди.}{Prjesvjat\'{a}ja Tr\'{o}icje, pom\'{i}luj nas; G\'{o}spodi, oč\'{i}sti grjeh\'{i} n\'{a}ša; Vlad\'{y}ko, prost\'{i} bjezzak\'{o}nja n\'{a}ša; Svjat\'{y}j, posjet\'{i} i iscjel\'{i} n\'{je}moŝi n\'{a}ša, \'{i}mjenje Tvo\'{je}go r\'{a}di.}{O most holy Trinity, have mercy on us. O Lord, cleanse us from our sins. O Master, pardon our transgressions. O Holy One, visit and heal our infirmities for Thy name's sake.}
\IbidSings{Г\'{о}споди, пом\'{и}луй.\LITMOD{(3x)}}{G\'{o}spodi, pom\'{i}luj.\LITMOD{(3x)}}{Lord, have mercy.\LITMOD{(3x)}}
\IbidSings{Сл\'{а}ва Отц\'{у} и С\'{ы}ну и Свят\'{о}му Д\'{у}ху, и н\'{ы}не и пр\'{и}сно и во в\'{е}ки век\'{о}в. Ам\'{и}нь.}{Sl\'{a}va Otc\'{u} i S\'{y}nu i Svjat\'{o}mu D\'{u}hu, i n\'{y}nje i pr\'{i}sno i vo v\'{je}ki vjek\'{o}v. Am\'{i}n.}{Glory to the Father, and to the Son, and to the Holy Spirit, now and ever and unto ages of ages. Amen.}
\IbidSings{\'{О}тче наш, \'{И}же ес\'{и} на небес\'{е}х! Да свят\'{и}тся \'{и}мя Тво\'{е}, да при\'{и}дет Ц\'{а}рствие Тво\'{е}, да б\'{у}дет в\'{о}ля Тво\'{я}, \'{я}ко на небес\'{и} и на земл\'{и}. Хлеб наш нас\'{у}щный д\'{а}ждь нам днесь; и ост\'{а}ви нам д\'{о}лги н\'{а}ша, \'{я}коже и мы оставл\'{я}ем должник\'{о}м н\'{а}шим; и не введ\'{и} нас во искуш\'{е}ние, но изб\'{а}ви нас от лук\'{а}ваго.}{\'{O}tčje naš, \'{I}žje jes\'{i} na njebjes\'{je}h! Da svjat\'{i}tsja \'{i}mja Tvo\'{je}, da pri\'{i}djet C\'{a}rstvije Tvo\'{je}, da b\'{u}djet v\'{o}lja Tvo\'{ja}, \'{ja}ko na njebjes\'{i} i na zjeml\'{i}. Hljeb naš nas\'{u}ŝnyj d\'{a}žd nam dnjes; i ost\'{a}vi nam d\'{o}lgi n\'{a}ša, \'{ja}kožje i my ostavl\'{ja}jem dolžnik\'{o}m n\'{a}šim; i nje vvjed\'{i} nas vo iskuš\'{je}nije, no izb\'{a}vi nas ot luk\'{a}vago.}{Our Father, Which art in the Heavens, hallowed be Thy Name. Thy Kingdom come. Thy will be done, on earth as it is in Heaven. Give us this day our daily bread. And forgive us our debts, as we forgive our debtors. And lead us not into temptation, but deliver us from the evil one.}
\PriestSays{Яко Тво\'{е} есть Ц\'{а}рство и с\'{и}ла и сл\'{а}ва Отц\'{а} и С\'{ы}на и Свят\'{а}го Д\'{у}ха, н\'{ы}не и пр\'{и}сно и во веки век\'{о}в.}{Jako Tvo\'{je} jest C\'{a}rstvo i s\'{i}la i sl\'{a}va Otc\'{a} i S\'{y}na i Svjat\'{a}go D\'{u}ha, n\'{y}nje i pr\'{i}sno i vo vjeki vjek\'{o}v.}{For Thine is the kingdom, and the power, and the glory: of the Father, and of the Son, and of the Holy Spirit, now and ever and unto ages of ages.}
\ReaderSays{Ам\'{и}нь.}{Am\'{i}n.}{Amen.}



\LiturgyComment{}{}{The reader says the appropriate tropar and kondak:}
\subsubsection{Liturgy of St. Chrysostom}
\LiturgyHeader{Тропарь святому Иоанну Златоустому (Глас 8)}{Tropar svjatomu Ioannu Zlatoustomu (Glas 8)}{Tropar to Saint John Chrysostom (Tone 8)}
\IbidSings{Уст тво\'{и}х, \'{я}коже св\'{е}тлость огн\'{я}, / возси\'{я}вши благод\'{а}ть, всел\'{е}нную просвет\'{и}; / не среброл\'{ю}бия м\'{и}рови сокр\'{о}вища сниск\'{а}, / высот\'{у} нам смиренном\'{у}дрия показ\'{а}, / но тво\'{и}ми словес\'{ы} наказ\'{у}я, \'{о}тче Ио\'{а}нне Злато\'{у}сте, / мол\'{и} Сл\'{о}ва Христ\'{а} Б\'{о}га спаст\'{и}ся душ\'{а}м н\'{а}шим.}{Ust tvo\'{i}h, \'{ja}kožje sv\'{je}tlost ogn\'{ja}, / vozsi\'{ja}vši blagod\'{a}t, vsjel\'{je}nnuju prosvjet\'{i}; / nje srjebrol\'{ju}bja m\'{i}rovi sokr\'{o}viŝa snisk\'{a}, / vysot\'{u} nam smirjennom\'{u}drja pokaz\'{a}, / no tvo\'{i}mi slovjes\'{y} nakaz\'{u}ja, \'{o}tčje Io\'{a}nnje Zlato\'{u}stje, / mol\'{i} Sl\'{o}va Hrist\'{a} B\'{o}ga spast\'{i}sja duš\'{a}m n\'{a}šim.}{Grace shining forth from your lips like a beacon has enlightened the universe. It has shown to the world the riches of poverty. It has revealed to us the heights of humility. Teaching us by your words, O father John Chrysostom, intercede before the Word, Christ our God, to save our souls.}
%\LiturgyComment{}{}{or}
%\LiturgyHeader{Тропарь Святому Василию (Глас 1)}{}{Tropar to Saint Basil (Tone 1)}
%\IbidSings{Во всю з\'{е}млю из\'{ы}де вещ\'{а}ние тво\'{е},/ \'{я}ко при\'{е}мшую сл\'{о}во тво\'{е},/ \'{и}мже богол\'{е}пно науч\'{и}л ес\'{и},/ естеств\'{о} с\'{у}щих уясн\'{и}л ес\'{и},/ челов\'{е}ческия об\'{ы}чаи украс\'{и}л ес\'{и},/ ц\'{а}рское свящ\'{е}ние, \'{о}тче препод\'{о}бне,/ мол\'{и} Христ\'{а} Б\'{о}га// спаст\'{и}ся душ\'{а}м н\'{а}шим.}{}{Your proclamation has gone out into all the earth / Which was divinely taught by hearing your voice / Expounding the nature of creatures, / Ennobling the manners of men. / O holy father of a royal priesthood, / Entreat Christ God that our souls may be saved.} %SOURCE: https://oca.org/saints/troparia/2018/01/01/100003-st-basil-the-great-archbishop-of-csarea-in-cappadocia
%\LiturgyComment{}{}{or}
%\LiturgyHeader{Тропарь Святому Григорию (Глас 4)}{}{Tropar to Saint Gregory (Tone 4)}
%\IbidSings{\'{И}же от Б\'{о}га св\'{ы}ше Бож\'{е}ственную благод\'{а}ть воспри\'{е}м, / сл\'{а}вне Григ\'{о}рие, / и Тог\'{о} с\'{и}лою укрепл\'{я}емь, / ев\'{а}нгельски ш\'{е}ствовати изв\'{о}лил ес\'{и}. / Отон\'{у}дуже у Христ\'{а} возм\'{е}здие труд\'{о}в при\'{я}л ес\'{и}, всеблаж\'{е}нне: / Ег\'{о}же мол\'{и}, да спас\'{е}т д\'{у}ши н\'{а}ша.}{}{Receiving divine grace from God on high, glorious Gregory, / and strengthened with its power, / you willed to walk in the path of the Gospel, most blessed one. / Therefore you have received from Christ the reward of your labors. / Entreat Him that He may save our souls.} %SOURCE: https://oca.org/saints/troparia/0216/03/12/100789-st-gregory-dialogus-the-pope-of-rome
\IbidSings{Сл\'{а}ва Отц\'{у} и С\'{ы}ну и Свят\'{о}му Д\'{у}ху.}{Sl\'{a}va Otc\'{u} i S\'{y}nu i Svjat\'{o}mu D\'{u}hu.}{Glory to the Father, and to the Son, and to the Holy Spirit.}
%\LiturgyComment{}{}{The reader says the appropriate kondak:}
\LiturgyHeader{Кондак святому Иоанну Златоустому (Глас 6)}{Kondak svjatomu Ioannu Zlatoustomu (Glas 6)}{Kondak to Saint John Chrysostom (Tone 6)}
\IbidSings{От неб\'{е}с при\'{я}л ес\'{и} Бож\'{е}ственную благод\'{а}ть, / и тво\'{и}ми устн\'{а}ми вся уч\'{и}ши / поклан\'{я}тися в Тр\'{о}ице Ед\'{и}ному Б\'{о}гу, / Ио\'{а}нне Злато\'{у}сте всеблаж\'{е}нне, препод\'{о}бне, дост\'{о}йно хв\'{а}лим тя: / ес\'{и} бо наст\'{а}вник, \'{я}ко Бож\'{е}ственная явл\'{я}я.}{Ot njeb\'{je}s pri\'{ja}l jes\'{i} Bož\'{je}stvjennuju blagod\'{a}t, / i tvo\'{i}mi ustn\'{a}mi vsja uč\'{i}ši / poklan\'{ja}tisja v Tr\'{o}icje Jed\'{i}nomu B\'{o}gu, / Io\'{a}nnje Zlato\'{u}stje vsjeblaž\'{je}nnje, prjepod\'{o}bnje, dost\'{o}jno hv\'{a}lim tja: / jes\'{i} bo nast\'{a}vnik, \'{ja}ko Bož\'{je}stvjennaja javl\'{ja}ja.}{From heaven you received the grace of God, teaching us by your words to worship the one God in Trinity. We worthily praise you, O blessed John Chrysostom, well pleasing to God, for you are a teacher revealing things divine.}
%\LiturgyComment{}{}{or}
%\LiturgyHeader{Кондак Святому Василию (Глас 4)}{}{Kondak to Saint Basil (Tone 4)}
%\IbidSings{Яв\'{и}лся ес\'{и} основ\'{а}ние непоколеб\'{и}мое Ц\'{е}ркве,/ пода\'{я} всем некрад\'{о}мое госп\'{о}дьство челов\'{е}ком,/ запечатл\'{е}я тво\'{и}ми вел\'{е}ньми,// небоявл\'{е}нне Вас\'{и}лие препод\'{о}бне.}{}{You were revealed as the sure foundation of the Church, / granting all mankind a lordship which cannot be taken away, / sealing it with your precepts, / venerable Basil, revealer of heaven.} %SOURCE: https://oca.org/saints/troparia/2018/01/01/100003-st-basil-the-great-archbishop-of-csarea-in-cappadocia
% Alternate version: You were revealed as the sure foundation of the Church, / Granting all men a lordship which cannot be taken away, / Sealing it with your precepts, / O Venerable and Heavenly Father Basil.
%\LiturgyComment{}{}{or}
%\LiturgyHeader{Кондак Святому Григорию (Глас 3)}{}{Kondak to Saint Gregory (Tone 3)}
%\IbidSings{Подобонач\'{а}льник показ\'{а}лся ес\'{и} Нач\'{а}льника п\'{а}стырем Хри\-ст\'{а}, / \'{и}ноков чред\'{ы}, \'{о}тче Григ\'{о}рие, / ко огр\'{а}де неб\'{е}сней наставл\'{я}я, / и отт\'{у}ду науч\'{и}л ес\'{и} ст\'{а}до Христ\'{о}во з\'{а}поведем Ег\'{о}: / н\'{ы}не же с н\'{и}ми р\'{а}дуешися, / и лик\'{у}еши в неб\'{е}сных кр\'{о}вех.}{}{Father Gregory, you showed yourself to be an imitator of Christ, the chief Shepherd, / guiding the orders of monks to the fold of heaven. / You taught the flock of Christ His commandments. / Now you rejoice and dance with them in the mansions of heaven.} %SOURCE: https://oca.org/saints/troparia/0216/03/12/100789-st-gregory-dialogus-the-pope-of-rome
\IbidSings{И н\'{ы}не и пр\'{и}сно и во в\'{е}ки век\'{о}в. Ам\'{и}нь.}{I Svjat\'{o}mu D\'{u}hu, i n\'{y}nje i pr\'{i}sno i vo v\'{je}ki vjek\'{o}v. Am\'{i}n.}{Now and ever and unto ages of ages. Amen.}


\subsubsection{Liturgy of St. Basil}
\LiturgyHeader{Тропарь Святому Василию (Глас 1)}{Tropar Svjatomu Vasiliju (Glas 1)}{Tropar to Saint Basil (Tone 1)}
\IbidSings{Во всю з\'{е}млю из\'{ы}де вещ\'{а}ние тво\'{е},/ \'{я}ко при\'{е}мшую сл\'{о}во тво\'{е},/ \'{и}мже богол\'{е}пно науч\'{и}л ес\'{и},/ естеств\'{о} с\'{у}щих уясн\'{и}л ес\'{и},/ челов\'{е}ческия об\'{ы}чаи украс\'{и}л ес\'{и},/ ц\'{а}рское свящ\'{е}ние, \'{о}тче препод\'{о}бне,/ мол\'{и} Христ\'{а} Б\'{о}га// спаст\'{и}ся душ\'{а}м н\'{а}шим.}{Vo vsju z\'{je}mlju iz\'{y}dje vjeŝ\'{a}nije tvo\'{je},/ \'{ja}ko pri\'{je}mšuju sl\'{o}vo tvo\'{je},/ \'{i}mžje bogol\'{je}pno nauč\'{i}l jes\'{i},/ jestjestv\'{o} s\'{u}ŝih ujasn\'{i}l jes\'{i},/ čjelov\'{je}čjeskja ob\'{y}čai ukras\'{i}l jes\'{i},/ c\'{a}rskoje svjaŝ\'{je}nije, \'{o}tčje prjepod\'{o}bnje,/ mol\'{i} Hrist\'{a} B\'{o}ga// spast\'{i}sja duš\'{a}m n\'{a}šim.}{Your proclamation has gone out into all the earth / Which was divinely taught by hearing your voice / Expounding the nature of creatures, / Ennobling the manners of men. / O holy father of a royal priesthood, / Entreat Christ God that our souls may be saved.} %SOURCE: https://oca.org/saints/troparia/2018/01/01/100003-st-basil-the-great-archbishop-of-csarea-in-cappadocia
\IbidSings{Сл\'{а}ва Отц\'{у} и С\'{ы}ну и Свят\'{о}му Д\'{у}ху.}{Sl\'{a}va Otc\'{u} i S\'{y}nu i Svjat\'{o}mu D\'{u}hu.}{Glory to the Father, and to the Son, and to the Holy Spirit.}
\LiturgyHeader{Кондак Святому Василию (Глас 4)}{Kondak Svjatomu Vasiliju (Glas 4)}{Kondak to Saint Basil (Tone 4)}
\IbidSings{Яв\'{и}лся ес\'{и} основ\'{а}ние непоколеб\'{и}мое Ц\'{е}ркве,/ пода\'{я} всем некрад\'{о}мое госп\'{о}дьство челов\'{е}ком,/ запечатл\'{е}я тво\'{и}ми вел\'{е}ньми,// небоявл\'{е}нне Вас\'{и}лие препод\'{о}бне.}{Jav\'{i}lsja jes\'{i} osnov\'{a}nije njepokoljeb\'{i}moje C\'{je}rkvje,/ poda\'{ja} vsjem njekrad\'{o}moje gosp\'{o}dstvo čjelov\'{je}kom,/ zapječatl\'{je}ja tvo\'{i}mi vjel\'{je}nmi,// njebojavl\'{je}nnje Vas\'{i}lije prjepod\'{o}bnje.}{You were revealed as the sure foundation of the Church, / granting all mankind a lordship which cannot be taken away, / sealing it with your precepts, / venerable Basil, revealer of heaven.} %SOURCE: https://oca.org/saints/troparia/2018/01/01/100003-st-basil-the-great-archbishop-of-csarea-in-cappadocia
\IbidSings{И н\'{ы}не и пр\'{и}сно и во в\'{е}ки век\'{о}в. Ам\'{и}нь.}{I Svjat\'{o}mu D\'{u}hu, i n\'{y}nje i pr\'{i}sno i vo v\'{je}ki vjek\'{o}v. Am\'{i}n.}{Now and ever and unto ages of ages. Amen.}

\subsubsection{Liturgy of St. Gregory}
\LiturgyHeader{Тропарь Святому Григорию (Глас 4)}{Tropar Svjatomu Grigoriju (Glas 4)}{Tropar to Saint Gregory (Tone 4)}
\IbidSings{\'{И}же от Б\'{о}га св\'{ы}ше Бож\'{е}ственную благод\'{а}ть воспри\'{е}м, / сл\'{а}вне Григ\'{о}рие, / и Тог\'{о} с\'{и}лою укрепл\'{я}емь, / ев\'{а}нгельски ш\'{е}ствовати изв\'{о}лил ес\'{и}. / Отон\'{у}дуже у Христ\'{а} возм\'{е}здие труд\'{о}в при\'{я}л ес\'{и}, всеблаж\'{е}нне: / Ег\'{о}же мол\'{и}, да спас\'{е}т д\'{у}ши н\'{а}ша.}{\'{I}žje ot B\'{o}ga sv\'{y}šje Bož\'{je}stvjennuju blagod\'{a}t vospri\'{je}m, / sl\'{a}vnje Grig\'{o}rije, / i Tog\'{o} s\'{i}loju ukrjepl\'{ja}jem, / jev\'{a}ngjelski š\'{je}stvovati izv\'{o}lil jes\'{i}. / Oton\'{u}dužje u Hrist\'{a} vozm\'{je}zdije trud\'{o}v pri\'{ja}l jes\'{i}, vsjeblaž\'{je}nnje: / Jeg\'{o}žje mol\'{i}, da spas\'{je}t d\'{u}ši n\'{a}ša.}{Receiving divine grace from God on high, glorious Gregory, / and strengthened with its power, / you willed to walk in the path of the Gospel, most blessed one. / Therefore you have received from Christ the reward of your labors. / Entreat Him that He may save our souls.} %SOURCE: https://oca.org/saints/troparia/0216/03/12/100789-st-gregory-dialogus-the-pope-of-rome
\IbidSings{Сл\'{а}ва Отц\'{у} и С\'{ы}ну и Свят\'{о}му Д\'{у}ху.}{Sl\'{a}va Otc\'{u} i S\'{y}nu i Svjat\'{o}mu D\'{u}hu.}{Glory to the Father, and to the Son, and to the Holy Spirit.}
% Alternate version: You were revealed as the sure foundation of the Church, / Granting all men a lordship which cannot be taken away, / Sealing it with your precepts, / O Venerable and Heavenly Father Basil.
\LiturgyHeader{Кондак Святому Григорию (Глас 3)}{Kondak Svjatomu Grigoriju (Glas 3)}{Kondak to Saint Gregory (Tone 3)}
\IbidSings{Подобонач\'{а}льник показ\'{а}лся ес\'{и} Нач\'{а}льника п\'{а}стырем Хри\-ст\'{а}, / \'{и}ноков чред\'{ы}, \'{о}тче Григ\'{о}рие, / ко огр\'{а}де неб\'{е}сней наставл\'{я}я, / и отт\'{у}ду науч\'{и}л ес\'{и} ст\'{а}до Христ\'{о}во з\'{а}поведем Ег\'{о}: / н\'{ы}не же с н\'{и}ми р\'{а}дуешися, / и лик\'{у}еши в неб\'{е}сных кр\'{о}вех.}{Podobonač\'{a}lnik pokaz\'{a}lsja jes\'{i} Nač\'{a}lnika p\'{a}styrjem Hri\-st\'{a}, / \'{i}nokov črjed\'{y}, \'{o}tčje Grig\'{o}rije, / ko ogr\'{a}dje njeb\'{je}snjej nastavl\'{ja}ja, / i ott\'{u}du nauč\'{i}l jes\'{i} st\'{a}do Hrist\'{o}vo z\'{a}povjedjem Jeg\'{o}: / n\'{y}nje žje s n\'{i}mi r\'{a}duješisja, / i lik\'{u}ješi v njeb\'{je}snyh kr\'{o}vjeh.}{Father Gregory, you showed yourself to be an imitator of Christ, the chief Shepherd, / guiding the orders of monks to the fold of heaven. / You taught the flock of Christ His commandments. / Now you rejoice and dance with them in the mansions of heaven.} %SOURCE: https://oca.org/saints/troparia/0216/03/12/100789-st-gregory-dialogus-the-pope-of-rome
\IbidSings{И н\'{ы}не и пр\'{и}сно и во в\'{е}ки век\'{о}в. Ам\'{и}нь.}{I Svjat\'{o}mu D\'{u}hu, i n\'{y}nje i pr\'{i}sno i vo v\'{je}ki vjek\'{o}v. Am\'{i}n.}{Now and ever and unto ages of ages. Amen.}



\subsection{Theotokion}
\IbidSings{Предст\'{а}тельство христи\'{а}н непост\'{ы}дное,/ ход\'{а}тайство ко Творц\'{у} непрел\'{о}жное,/ не пр\'{е}зри гр\'{е}шных мол\'{е}ний гл\'{а}сы,/ но предвар\'{и}, \'{я}ко Благ\'{а}я,/ на п\'{о}мощь нас, в\'{е}рно зов\'{у}щих Ти;/ ускор\'{и} на мол\'{и}тву и потщ\'{и}ся на умол\'{е}ние,// предст\'{а}\-тельствующи пр\'{и}сно,/ Богор\'{о}дице, чт\'{у}щих Тя.}{Prjedst\'{a}tjelstvo hristi\'{a}n njepost\'{y}dnoje,/ hod\'{a}tajstvo ko Tvor\-c\'{u} njeprjel\'{o}žnoje,/ nje pr\'{je}zri gr\'{je}šnyh mol\'{je}nij gl\'{a}sy,/ no prjedvar\'{i}, \'{ja}ko Blag\'{a}ja,/ na p\'{o}moŝ nas, v\'{je}rno zov\'{u}ŝih Ti;/ uskor\'{i} na mol\'{i}tvu i potŝ\'{i}sja na umol\'{je}nije,// prjedst\'{a}\-tjelst\-vujuŝi pr\'{i}sno,/ Bogor\'{o}dicje, čt\'{u}ŝih Tja.}{O Protection of Christians that cannot be put to shame, unchanging Mediation before the Creator, despise not the voice of the sinners' prayer, but in that you are good, come quickly to help us who call upon you with faith. Make speed to intercede and make haste to supplicate, O Theotokos, who ever protect those who honour you.}
\IbidSings{Г\'{о}споди, пом\'{и}луй. \LITMOD{(12x)}}{G\'{o}spodi, pom\'{i}luj. \LITMOD{(12x)}}{Lord, have mercy. \LITMOD{(12x)}}
\IbidSings{Сл\'{а}ва Отц\'{у} и С\'{ы}ну и Свят\'{о}му Д\'{у}ху, и н\'{ы}не и пр\'{и}сно и во в\'{е}ки век\'{о}в. Ам\'{и}нь.}{Sl\'{a}va Otc\'{u} i S\'{y}nu i Svjat\'{o}mu D\'{u}hu, i n\'{y}nje i pr\'{i}sno i vo v\'{je}ki vjek\'{o}v. Am\'{i}n.}{Glory to the Father, and to the Son, and to the Holy Spirit, now and ever and unto ages of ages. Amen.}
%TODO: is the translation consistent?
\IbidSings{Честн\'{е}йшую Херув\'{и}м и Сл\'{а}внейшую без сравн\'{е}ния Сераф\'{и}м, без истл\'{е}ния Б\'{о}га Сл\'{о}ва р\'{о}ждшую, с\'{у}щую Богор\'{о}дицу, Тя велич\'{а}ем.}{Čjestn\'{je}jšuju Hjeruv\'{i}m i Sl\'{a}vnjejšuju bjez sravn\'{je}nja Sjeraf\'{i}m, bjez istl\'{je}nja B\'{o}ga Sl\'{o}va r\'{o}ždšuju, s\'{u}ŝuju Bogor\'{o}dicu, Tja vjelič\'{a}jem.}{More honourable than the Cherubim, and more glorious beyond compare than the Seraphim: without defilement you gave birth to God the Word: true Theotokos, we magnify you.}
\IbidSings{\'{И}менем Госп\'{о}дним благослов\'{и}, \'{о}тче!}{\'{I}mjenjem Gosp\'{o}dnim blagoslov\'{i}, \'{o}tčje!}{In the name of the Lord, father bless.}
%TODO: add translation (is this even correct?)
%Молитвама Светих Отаца наших, Господе Исусе Христе Сине Божији, помилуј нас грешне.
\PriestSays{Молитвами святых отец наших, Г\'{о}споди Иисусе Христе Боже наш, помилуй нас. Ам\'{и}нь.}{}{Through the prayers of our holy fathers, Lord Jesus Christ our God, have mercy on us and save us.}
\ChoirSays{Ам\'{и}нь.}{Amin.}{Amen.}
\begin{comment}
\clearpage

%\cleardoublepage
\clearpage
\section{Special Hymns}
%TODO: mention that this is for the 3rd Sunday of Lent
\begin{HymnPartPage}\lilypondfile[quote, noindent, line-width=8.000000\in]{HYMNS_DIRECTORY/Cross_of_the_Lord.sub.ly}\label{Cross_of_the_Lord}\end{HymnPartPage}

%TODO: include a reference for each of these
%TODO: mention that this is for the 3rd Sunday of Lent
\HymnFullPage{To_Thy_Cross}

%TODO: mention that this replaces "O tebje..." on Annunciation day
\HymnFullPage{Zadostoinik_Annunciation}

\HymnFullPage{V_Nedelu_Vani} % a hymn for Palm Sunday
\HymnFullPage{Prichasten} % a communion hymn for Palm Sunday

%\HymnFullPage{Zadostoinik_Pascha} % a hymn for Pascha

\HymnFullPage{Da_Molchit}
\HymnFullPage{Voskreseniye_Tvoye}

\HymnFullPage{Pentecostal_Week}
\HymnFullPage{Zadostoinik_Pentecost}

\HymnFullPage{Ascension_hymn}

\HymnFullPage{Flesh_Asleep}

%\HymnFullPage{Great_Litany}

%TODO: this hymn does not work without Latin lyrics
\HymnFullPage{Ton_Despotin}

\HymnFullPage{Vjechnaya_Pamjat}
\HymnFullPage{Memorial}
\HymnFullPage{blazheni}
\HymnFullPage{Memorial_Kondak}

%\clearpage
%\section{Variables}
%\subsection{Litany for the Departed}
%TODO: Ектения о усопших
%\label{Departed}
\end{comment}

\end{document}

\begin{comment}
\LITNOTE{Or, during a Liturgy of the Presanctified Gifts:}

\LITTITLE{[``Now the powers of heaven\ldots'' - page~\pageref{Nynje_sily_nebsnyja}]}\\

\begin{paracol}{2}
{
\selectlanguage{russian}
\LITSPK{Лик:}
Ныне силы небесныя
снами невидимо служат.
Се бо входит, входит Царь Славы,
се жертва тайная совершенна дориносится.

Верою и любовию приступим да причастницы жизнивечныя будем.

Аллилуйа \LITMOD{(3x)}
}
\switchcolumn
\LITSPK{Choir:}

Now the powers of heaven do serve invisibly with us.
Lo, the King of Glory enters.
Lo, the mystical sacrifice is upborne, fulfilled.

Let us draw near in faith and love, and become communicants of life eternal.

Alliluia \LITMOD{(3x)}
\end{paracol}
\end{comment}

